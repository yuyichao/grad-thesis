% -*- mode: latex-mode; TeX-engine: xetex; LaTeX-command-style: (("" "SOURCE_DATE_EPOCH=0 %(PDF)%(latex) --shell-escape %S%(PDFout)")); TeX-master: "../dissertation.tex"; -*-

\chapter{Photoassociation of Single Atoms}
\label{ch:pa}

\section{Introduction}

\todo{
  Need to measure the excited states in order to drive a two photon transition
  through the excited state.
}

\section{Energy Levels}

First we will discuss the energy levels in a diatomic molecule
as well as the labeling system for the states.
We will focus mainly on the electronic excited states measured in this chapter
but most of the discussion here applies to ground electronic states as well
and will be useful for chapter \label{ch:raman-spectroscopy} and \label{ch:raman-transfer}.

\subsection{Angular Momentums}

Compared to an atom, a diatomic molecule has many more degrees of freedoms.
In additional to the quantum numbers for each atom in the molecule,
molecules also have nuclear motion.
In order to reduce the complexity, it is therefore very important to consider the
symmetry of the system, and in particular the angular momentums,
which corresponds to rotation symmetry, and the coupling between them.
The angular momentums in a diatomic molecule includes electron orbit $\mathbf{L}$
\footnote{There are $\mathbf{L}_1$ and $\mathbf{L}_2$ for the two electron but since
  we only consider states with at most one $\mathbf{L}_i\neq0$ we will only use one quantum number here},
electron spin $\mathbf{S}_1$ and $\mathbf{S}_2$, nuclear orbital $\mathbf{N}$
and nuclear spin $\mathbf{I}_1$ and $\mathbf{I}_2$.
Although the total angular momentum
$\mathbf{F}\equiv\mathbf{L}+\mathbf{S}_1+\mathbf{S}_2+\mathbf{N}+\mathbf{I}_1+\mathbf{I}_2$
is the only true conserved quantity in the absence of external field,
depending on the coupling strengths between the angular momentums,
there are additional approximately conserved quantity in the molecule.

For the NaCs molecule and our experiment, there are two important regimes where the coupling
strength can be easily ordered.

\subsubsection{Deeply Bound States}

\begin{figure}
  \centering
  \includegraphics[width=\textwidth]{figures/pa_hunds_case_a.pdf}
  \caption[Hund's case (a)]{
    Angular momentum coupling for \textit{Hund's case (a)}.
    $\mathbf{L}$ and $\mathbf{S}$ are coupled to the internuclear axis $\hat n$
    and the sum of the projections $\Omega=\Lambda+\Sigma$ is then
    added with the orthogonal compoment $\mathbf{N}$ to form $\mathbf{J}$.
    \label{fig:pa-hunds-case-a}}
\end{figure}

This is described by the \textit{Hund's case (a)}.
Molecular states with large binding energies mostly experience interactions
between the atoms at short range where the electric static interaction is very strong.
This couples the the two electron spins into a total electron spin
$\mathbf{S}\equiv\mathbf{S}_1+\mathbf{S}_2$ via a very strong effective interaction
of the form $\mathbf{S}_1\cdot\mathbf{S}_2$ which originates
from the resulting symmetry of the electron orbital wavefunction.
Similar to atoms, the nuclear spin interaction is also very week compared to
other energy scales so we can ignore the hyperfine structure and only need to consider
$\mathbf{J}\equiv\mathbf{L}+\mathbf{S}+\mathbf{N}$.

The strong electrostatic interaction also creates an effective coupling
between the $\mathbf{L}$ and $\mathbf{S}$ with the internuclear axis $\hat{n}$
causing $\mathbf{L}$ and $\mathbf{S}$ to process rapidly around $\hat{n}$.
This creates two new conserved quantity $\Lambda$ and $\Sigma$
as the projection of $\mathbf{L}$ and $\mathbf{S}$
along $\hat{n}$ respectively.
The total angular momentum along $\hat{n}$ is therefore $\Omega\equiv\Lambda+\Sigma$
and it is added to the $\mathbf{N}$ which is orthogonal to $\hat{n}$ to form
the total angular momentum $\mathbf{J}$ (Fig~\ref{fig:pa-hunds-case-a}).

The angular momenum state of the molecule is therefore fully characterized by
$|L,\Lambda,S,\Omega,J\rangle$. $\Lambda$ can be $0,1,\dots,L$, $\Omega$ ranges from
$\abs{\Lambda-S}$ to $\Lambda+S$ and $J\geqslant\Omega$.
The $L$ quantum number is specified by the electronic state and will be discussed
in section \ref{ch:pa:pes} and the rest of the angular momentum quantum numbers
are represented by the \textit{Hund's case (a)} term symbol,
\[ ^{2S+1}\Lambda_\Omega \]
similar to the atomic term symbol $^{2S+1}L_J$.
Just as the use of capital English letters $S,P,D,\dots$ to represent
$L=0,1,2,\dots$, capital Greek letters $\Sigma,\Pi,\Delta,\dots$ are used
to denote $\Lambda=0,1,2,\dots$ in the term symbol.
An additional symmetry to consider is the reflection about a plane that includes
the internuclear axis.
For $\Lambda>0$ states, the reflection produces a new state at the same energy
creating the so-called $\Lambda$-doubling. For $\Lambda=0$ states, i.e. $\Sigma$ states,
the reflection produces the same state with a phase of $\pm1$.
This phase is also included in the term symbol to fully specify the symmetry of
a $\Sigma$ states as
\[ ^{2S+1}\Sigma_\Omega^{\pm} \]
Note the $\Sigma$ state here should not be confused with the quantum number $\Sigma$.

\todo{rotational energy? J(J+1) - Omega^2}

\subsubsection{Near Threshold Bound States}

For molecular states with small binding energy, the interaction between the two atoms is
small compared to the internal coupling in the atoms and
the angular momentum coupling is ``atom like''.
In this limit, the total angular momentum $\mathbf{F}_1$ and $\mathbf{F}_2$
for the individual atoms forms $\mathbf{F}_{atom}=\mathbf{F}_1+\mathbf{F}_2$
which is then coupled to the nuclear rotation $\mathbf{N}$
to form $\mathbf{F}=\mathbf{F}_{atom}+\mathbf{N}$.

\todo{mention N=2 spectroscopy?}

\subsection{Potential Energy Surface}
\label{ch:pa:pes}
\todo{
  {
  \item Electronic (including spins via exchange interaction)
  \item Vibrational
  \item Rotations (coupled to the spin)
  \item HF, will be ignored for the most part for the excited state
    We will breifly discuss the ground state HF structure in next chapter
  }
  {
  \item BO approximation, motivated by the mass ratio between the nuclear and electron mass.
    Allows us to treat the electronic energy as a function of nuclear motion (PEC).
    Formally the approximation corresponds to the energy gap between electronic state
    to be much larger than neuclear motion, thus the approximation breaks down when
    electronic state energy crosses.
    For most of our discussion, we operate far away from the crossing of the electronic
    potentials and therefore the approximation is justified.
  }
  \todo{FCF}
}

\section{Effect of the Trap}

(light shift, broadening)

\section{Photoassociation Spectroscopy}
(v=0, 12, 14, etc)
