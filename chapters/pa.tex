% -*- mode: latex-mode; TeX-engine: xetex; LaTeX-command-style: (("" "SOURCE_DATE_EPOCH=0 %(PDF)%(latex) --shell-escape %S%(PDFout)")); TeX-master: "../dissertation.tex"; -*-

\chapter{Photoassociation of Single Atoms}
\label{ch:pa}

\section{Introduction}

\todo{
  Need to measure the excited states in order to drive a two photon transition
  through the excited state.
}

\section{Energy Levels}

First we will discuss the energy levels in a diatomic molecule as well as the labeling system.
We will focus mainly on the electronic excited states measured in this chapter
but most of the discussion here applies to ground electronic states as well
and will be useful for chapter \label{ch:raman-spectroscopy} and \label{ch:raman-transfer}.

\todo{FCF}

\todo{
\item Also mostly about the interaction at short range/state with large binding energy.

  {
  \item Has many degrees of freedoms (in additional to twice the number of electron orbit,
    electron spin and neuclear spin, also neuclear motion)
  \item Important to consider the symmetry, especially the angular momentum
  \item Strongest from S1-S2
  \item LS coupling, hunds case a
  }



  {
  \item Electronic (including spins via exchange interaction)
  \item Vibrational
  \item Rotations (coupled to the spin)
  \item HF, will be ignored for the most part for the excited state
    We will breifly discuss the ground state HF structure in next chapter
  }
  {
  \item BO approximation, motivated by the mass ratio between the nuclear and electron mass.
    Allows us to treat the electronic energy as a function of nuclear motion (PEC).
    Formally the approximation corresponds to the energy gap between electronic state
    to be much larger than neuclear motion, thus the approximation breaks down when
    electronic state energy crosses.
    For most of our discussion, we operate far away from the crossing of the electronic
    potentials and therefore the approximation is justified.
  }
}

\section{Effect of the Trap}

(light shift, broadening)

\section{Photoassociation Spectroscopy}
(v=0, 12, 14, etc)
