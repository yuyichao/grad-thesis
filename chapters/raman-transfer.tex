% -*- mode: latex-mode; TeX-engine: xetex; LaTeX-command-style: (("" "SOURCE_DATE_EPOCH=0 %(PDF)%(latex) --shell-escape %S%(PDFout)")); TeX-master: "../dissertation.tex"; -*-

\chapter{Coherent Optical Creation of NaCs Molecule}
\label{ch:raman-transfer}

\section{Introduction}

\todo{emphasize general approach}

\section{Raman Transition Beyond Three-Level Model}

In an ideal three-level system, the scattering probability during a $\pi$ pulse
Raman transition can be made arbitrarily small by using a large single photon detuning.
However, in a real system, there are often other effects that increases the scattering
and may also put a lower limit on the scattering probability during the transfer.
Other practical limitation in the system like stability of the laser power and frequency
also needs to be taken into account.
Fig.~\ref{fig:raman-transfer-generic-raman-model} shows a generic model
for a real Raman transition demostrating some of these effects.
In the experiment, we find the parameter range that gives the best transfer efficiency
using numerical simulation (see section \ref{ch:raman-transfer:state-selction}).
Nevertheless, in order to develop a general approach that can be applied to other systems,
it is also important to understand the various physical mechanism that leads
to the optimal parameters.
Therefore, in this section, we will discuss some of the most important effects
on the transfer efficiency at qualitative and semiquantitative level.
Due to experimental constraint, we will assume that the single photon detuning is
much smaller than the frequency of each individual beams, i.e. $\Delta\ll\nu_1,\ \nu_2$.

\begin{figure}
  \centering
  \includegraphics[width=0.6\textwidth]{figures/raman_transfer_generic_raman_model.pdf}
  \caption[Generic model for a real Raman transition]{
    Generic model for a real Raman transition.
    The initial state $|i\rangle$ and the final state $|f\rangle$
    has a energy difference $\delta$
    and are coupled by two Raman beams with frequencies and
    single photon Rabi frequencies of $\nu_1,\ \Omega_1$ and $\nu_2,\ \Omega_2$ respectively.
    The corresponding matrix elements (arbitrary unit) are $M_1$ and $M_2$.
    The Raman beams are detuned by $\Delta$ from the primary excited state $|e\rangle$,
    which has a decay rate of $\Gamma_e$.
    We also consider additional states near the initial ($|i'\rangle$),
    final ($|f'\rangle$) and intermediate excited $|e'\rangle$ states which are
    separated from the corresponding Raman transition states by $\omega'_i$,
    $\omega'_f$ and $\omega'_e$ respectively.
    Only one additional state of each kinds are included to simplify the discussion
    without loss of generality.
    \label{fig:raman-transfer-generic-raman-model}}
\end{figure}

\subsection{Additional Initial and Final States}

First, we will discuss the effect of $|i'\rangle$ and $|f'\rangle$ states
near the initial and final states.
These states can be coupled to the excited state $|e\rangle$ by the Raman beams,
which can in tern be coupled to the initial and final states
by an off-resonance Raman transition.
The leakage is suppressed by the detuning from the Raman resonance,
i.e. $\omega'_i$ and $\omega'_f$.
This puts a limit on the Raman Rabi frequency $\Omega_R$ to be smaller
than the smallest energy gap, which in turns puts a limit on the minimum Raman transfer time.
In our experiment, the minimum energy gap comes from axial motional excitation of
the atomic initial states which is between $2\pi\times10 - 30$~kHz
depending on the trap depth used.
The typical Raman $\pi$ time we can realize is $0.5 - 5$~ms so this effect
is not a major limiting factor for our transfer efficiency.

\subsection{Additional Excited states}

\begin{figure}
  \centering
  \includegraphics[width=\textwidth]{figures/raman_transfer_extra_ext_states.pdf}
  \caption[Raman transition with additional excited states]{
    Effect of additional excited states $|e'\rangle$ on the Raman transition efficiency.
    (A) Depending on the sign of the coupling, there could be constructive (blue)
    or destructive (orange) interference on the Raman Rabi frequency $\Omega_{Raman}$.
    (B) Increased scattering rate $\Gamma_{scatter}$ caused by $|e'\rangle$ with a minimal
    between the two states.
    (C) Optimal detunine exists between the two states with maximum transfer efficiency
    corresponds to a fraction of the state spacing.
    \label{fig:raman-transfer-extra-ext-states}}
\end{figure}

Next, we will consider the effect of the $|e'\rangle$ state near the excited intermediate state.
These states can be coupled to the ground states, both $|i\rangle$ and $|f\rangle$,
by the Raman beams and can cause a change in both the Raman Rabi frequency
and the scattering rate (Fig.~\ref{fig:raman-transfer-extra-ext-states}).

\subsection{Cross Coupling Between Light Addressing Initial and Final States}

\todo{Light shift?}
\todo{Footnote about assumption of matrix element being the same for two beams/polarization
  and general existance of other angular momentum states}

\section{Raman Transfer versus STIRAP}

An alternative method often used to create and prepare the internal states of ultracold molecule
is stimulated Raman adiabatic passage (STIRAP)\todo{\cite{}}.
Compared to Raman transition, which uses detuning from the excited state
to reduce scattering during the transfer, STIRAP relies on a superposition between
the initial and final state as a dark state to achieve the same goal.
The dark state in STIRAP is created due to a destructive interference of transition
from the initial and final state to the excited state.

Similar to Raman transfer, STIRAP in an ideal three-level system can achieve
full coherent transfer with arbitrarily small scattering probability
when given unlimited time and power budget.
However, in reality, states and coupling that exist outside the ideal three-level system
always have a non-zero probability of scattering loss.
In this section, we will apply the approach we took for Raman transition
and apply it to STIRAP. We will then compare the loss caused by different practical limitations
and discuss which approach should be taken under certain circumstance.

\subsection{Generic Model for STIRAP}

\ref{fig:raman-transfer-generic-stirap-model}

\begin{figure}
  \centering
  \includegraphics[width=0.6\textwidth]{figures/raman_transfer_generic_raman_model.pdf}
  \caption[Generic model for a real STIRAP]{
    \todo{}
    \label{fig:raman-transfer-generic-stirap-model}}
\end{figure}

\subsection{Additional Initial and Final States}

\subsection{Additional Excited states}

\subsection{Cross Coupling Between Light Addressing Initial and Final States}

\subsection{Conclusion}

\section{States Selection}
\label{ch:raman-transfer:state-selction}

(Differential Light Shift)
(Scattering)

\subsection{Excited State Selection}

\subsection{Ground States Selection}

\subsubsection{Final Molecular State}

\subsubsection{Initial Atomic State}

\section{Raman Transfer Results}

\subsection{Scaling of Raman Transition Parameters}
