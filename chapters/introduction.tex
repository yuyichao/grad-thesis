% -*- mode: latex-mode; TeX-engine: xetex; LaTeX-command-style: (("" "SOURCE_DATE_EPOCH=0 %(PDF)%(latex) --shell-escape %S%(PDFout)")); TeX-master: "../dissertation.tex"; -*-

\chapter{Introduction}
\label{ch:introduction}

\section{Ultracold Molecules}
\label{ch:introduction:molecules}

\todo{
  Starting from precise spectroscopy and laser cooling
  Thanks to high level of control, ultra neutral atoms has been .......
  Simple and weak interaction

  Molecules, rich internal degrees of freedom, larger interactions.
  Compared to ion, rydberg, longer live, tunable interaction
  (expand on the applications)

  The properties that makes molecules, however,
  makes cooling and controlling the molecules much harder than atoms.
  The additional degrees of freedom greatly increases the number of states
  to control and the lack of a cycling transition in general in molecules
  limits the number of photon one can scatter off of the molecules which
  affects laser cooling.
}

\todo{
  Association of molecules
  selection of NaCs molecules
}

\section{Optical Tweezers}
\label{ch:introduction:tweezers}
\todo{
  Magnetic trap, optical dipole trap, optical lattice (quantum microscope)
}

\section{Contents of this Thesis}
\label{ch:introduction:contents}

In this thesis, we describe the method we use to create a weakly bound ground state molecule
in the optical tweezer and the results leading up to it including the control of atoms
and the measurement on the interaction between the atoms and the molecular potential.

We start with chapter \ref{ch:computer-control} by giving a high level description of
the custom computer control system we use.
This system controls the timing of all the hardware outputs in the experiment
and is used to perform all the measurements in the rest of this thesis.

Chapter \ref{ch:loading} discusses the loading of the single atoms into the optical tweezer.
The discussion includes the preparation steps needed before the loading,
e.g. freespace cooling of atoms,
and the imaging of the atom in the tweezer,
which is the primary detection method used in our experiment.
All the experiments in the following chapters are performed using
the atoms and molecules in the optical tweezers.

Chapter \ref{ch:rsc} describes the Raman sideband cooling (RSC) process
we used to cool single Na atom in the optical tweezer.
As the lighter atom with a broader optical linewidth,
the RSC of Na atom faces additional challenges compared to atoms
that were cooled using similar method previously
and the tool we developed to overcome these challenges can be applied to other systems as well.

After preparation of the atomic state,
chapter \ref{ch:interaction-shift} starts our investigation of the interaction between atoms
by measuring the $s$-wave scattering length using interaction shift spectroscopy.
The measurement result is used to refine the prediction for Feshbach resonance
between Na and Cs atoms and also to improve the atomic state preparation.

Chapter \ref{ch:pa} discusses the measurement of the molecular excited states
by photoassociation spectroscopy.
We also study the effect of the tweezer beam on the molecular states and transitions.
The states mapped out in this measurement are used as intermediate states
to study the groud molecular states via two-photon transitions.

Following the previous chapter, chapter \ref{ch:raman-spectroscopy} discusses
the properties of the weakly bound ground molecular states using Raman spectroscopy.
We demostrated the ability to control the rotational states of the molecule
and studies the coupling of the molecule with external field
and the their hyperfine structure in the weakly bound regime.
The measurement identifies candidate target states for our coherent molecule creation.

Chapter \ref{ch:raman-transfer} combines the preparation and characterization
in the previous chapters and describes our all-optical coherent molecule formation process.
A detailed comparison between different approaches and states selection is given
to support our choice of experimental parameters.
We show our result on the coherent transfer and characterizes
the molecule we create as well as the transfer process.
We study the limit on the transfer efficiency
and list open questions in the transfer process
as well as possible future improvements.
