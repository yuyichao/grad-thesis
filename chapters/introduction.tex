% -*- mode: latex-mode; TeX-engine: xetex; LaTeX-command-style: (("" "SOURCE_DATE_EPOCH=0 %(PDF)%(latex) --shell-escape %S%(PDFout)")); TeX-master: "../dissertation.tex"; -*-

\chapter{Introduction}
\label{ch:introduction}

\section{Ultracold Molecules}
\label{ch:introduction:molecules}

Started with precise optical spectroscopy and the development of laser cooling
and trapping technologies, the increasing measurement precision and
high level of control has been one of the the primary driving forces
in the field of atomic physics in the past decades.
Systems based on cooling and controlling of neutral atoms
have been used in a wide range of applications including
quantum computing, quantum communication, precision measurement,
quantum simulation and the study of many other many body effects.

Despite these great successes, the utility of atom based systems
are limited by their simple internal structure and weak interactions.
Molecules, on the other hand, with its richer internal degrees of freedom
including electronic orbital, electron spin, neuclear vibration, rotation and spin,
the potential of stronger interaction and more variance of symmetry classes,
are better candidates for more applications especially in
quantum computing, precision measurement and quantum simulation.
Moreover, comparing to other systems with stronger interaction like ions and Rydberg atoms,
the interacting states in molecules are long lived and tunable thanks to
the abundance of low energy excitations,
which can offer better isolation from the environment giving rise to longer coherence time.

Many applications of molecules requires cooling and high level of control
on the quantum state of the molecules.
Unfortunately, the properties that makes molecules attractive
also makes controlling them harder.
The enabling technique for most ultracold atom experiment, laser cooling,
typically requires scattering of a large number of photons ($\approx10^4\sim10^7$).
This is possible in atomic system due to the existence of cycling transition
or near cycling ones that can be completely closed with
one or two spin states repumping lasers.
However, the lack of selection rules for vibrational states means that
such transition generally does not exist in molecules.
As a result, experiments aiming to achieve control of molecule on a similar level with atoms
usually take one of the two approaches,
\begin{enumerate}
\item Direct laser cooling of special molecules
  with approximate cycling transitions\todo{cite}.\\
  These are molecules that have optical transitions that has a high probability
  of decaying down to the same vibrational states.
  With the help of a few vibrational repumping lasers,
  scattering of $\approx10^3\sim10^5$ photons\todo{cite} can be achieved.
  Significant progress has been made using this approach in recent years
  including sub-doppler cooling and trapping of molecules in optical tweezers\todo{cite}.
  The main challenge with this approach is to achieve a better cooling performance
  given the still limitted photon scattering budget.
\item Creating of molecules from ultracold atoms.\\
  First realized a decade ago\todo{cite},
  this approach takes advantage of the mature cooling and trapping techniques
  developped for atoms and creates molecules from atoms
  that are cooled to ultracold temperature.
  The difficulty with this approach is understandably in the creation of the molecule,
  which must be done with acceptable efficiency and coherence
  in order to maintain the cooling and controling done on the atoms.
  This approach currently allows a lower temperature to be achieved
  and this is the approach used in our experiment.
\end{enumerate}

\section{Assembly of Molecules in Optical Tweezers}
\label{ch:introduction:tweezers}

\todo{
  Magnetic trap, optical dipole trap, optical lattice (quantum microscope)
  general approach
}

\todo{
  Overview of the experiment steps
}

\todo{
  selection of NaCs molecules
  Alkali atoms
  Large dipole moment
}

\section{Contents of this Thesis}
\label{ch:introduction:contents}

In this thesis, we describe the method we use to create a weakly bound ground state molecule
in the optical tweezer and the results leading up to it including the control of atoms
and the measurement on the interaction between the atoms and the molecular potential.

We start with chapter \ref{ch:computer-control} by giving a high level description of
the custom computer control system we use.
This system controls the timing of all the hardware outputs in the experiment
and is used to perform all the measurements in the rest of this thesis.

Chapter \ref{ch:loading} discusses the loading of the single atoms into the optical tweezer.
The discussion includes the preparation steps needed before the loading,
e.g. freespace cooling of atoms,
and the imaging of the atom in the tweezer,
which is the primary detection method used in our experiment.
All the experiments in the following chapters are performed using
the atoms and molecules in the optical tweezers.

Chapter \ref{ch:rsc} describes the Raman sideband cooling (RSC) process
we used to cool single Na atom in the optical tweezer.
As the lighter atom with a broader optical linewidth,
the RSC of Na atom faces additional challenges compared to atoms
that were cooled using similar method previously
and the tool we developed to overcome these challenges can be applied to other systems as well.

After preparation of the atomic state,
chapter \ref{ch:interaction-shift} starts our investigation of the interaction between atoms
by measuring the $s$-wave scattering length using interaction shift spectroscopy.
The measurement result is used to refine the prediction for Feshbach resonance
between Na and Cs atoms and also to improve the atomic state preparation.

Chapter \ref{ch:pa} discusses the measurement of the molecular excited states
by photoassociation spectroscopy.
We also study the effect of the tweezer beam on the molecular states and transitions.
The states mapped out in this measurement are used as intermediate states
to study the groud molecular states via two-photon transitions.

Following the previous chapter, chapter \ref{ch:raman-spectroscopy} discusses
the properties of the weakly bound ground molecular states using Raman spectroscopy.
We demostrated the ability to control the rotational states of the molecule
and studies the coupling of the molecule with external field
and the their hyperfine structure in the weakly bound regime.
The measurement identifies candidate target states for our coherent molecule creation.

Chapter \ref{ch:raman-transfer} combines the preparation and characterization
in the previous chapters and describes our all-optical coherent molecule formation process.
A detailed comparison between different approaches and states selection is given
to support our choice of experimental parameters.
We show our result on the coherent transfer and characterizes
the molecule we create as well as the transfer process.
We study the limit on the transfer efficiency
and list open questions in the transfer process
as well as possible future improvements.
