% -*- mode: latex-mode; TeX-engine: xetex; LaTeX-command-style: (("" "SOURCE_DATE_EPOCH=0 %(PDF)%(latex) --shell-escape %S%(PDFout)")); TeX-master: "../dissertation.tex"; -*-

\chapter{Interaction of Single Atoms}
\label{ch:interaction-shift}

\section{Scattering Length}
(Importance/relation with binding energy etc.)

\section{Two Interacting Atoms in Optical Tweezer}

The Hamiltonian for two atoms in an harmonic potential with interaction is,
\[
  H=\sum_{i=x,y,z}\paren{\frac{m_{1}\omega_{1,i}^2r_{1,i}^2}{2}+\frac{p_{1,i}^2}{2m_{1}}}+\sum_{i=x,y,z}\paren{\frac{m_{2}\omega_{2,i}^2r_{2,i}^2}{2}+\frac{p_{2,i}^2}{2m_{2}}}+V_{int}\paren{\mathbf{r}_1-\mathbf{r}_2}
\]
where $m_j$ is the mass of the $j$-th atom,
$r_{j,i}$, $p_{j,i}$, $\omega_{j,i}$ are the coordinate, momentum and trapping frequency
for the $j$-th atom along the $i$-th axis.
$V_{int}$ is the interaction potential between the two atoms which is only a function
of the relative coordinate between the atoms $\mathbf{r}_1-\mathbf{r}_2$.

Since the two atoms experience the same trapping light field,
their trapping potential has the same center and the same shape.
However, due to the difference in the polarizability between the atoms,
the trap depth can be different.
Nevertheless, in our experiment, depending on the trapping wavelength,
we have $\omega_{1,i}\approx\omega_{2,i}$ to within $10\%$ to $20\%$
and this is the regime we will mainly focus on in this section.

The interaction potential $V_{int}$ is the original for
the molecular bound states and its exact form will be discussed in chapter \ref{ch:pa},
\ref{ch:raman-spectroscopy} and \ref{ch:raman-transfer}.
However, since the range of the potential is much smaller than
the size of the atomic wavefunction, we can ignore the short range details of the potential
and treat it as a contact interaction characterized only by the scattering length $a$.
\[
  V_{int}\paren{\mathbf{r}}=\frac{2\pi\hbar^2a}{\mu}\delta_{reg}\paren{\mathbf{r}}
\]
where $\mu = m_1m_2/\paren{m_1+m_2}$ is the reduced mass and
$\delta_{reg}\paren{\mathbf{r}}\equiv\delta^{(3)}\paren{\mathbf{r}}\paren{\partial/\partial r}r$
is the regularized delta-function.

In order to calculate the interaction term, we can change from the coordinates for
the two individual atoms to the center of mass (COM) and relative coordinates.
\begin{align*}
  R_i=&\ \frac{m_1r_{1,i}+m_2r_{2,i}}{m_1+m_2}&r_{rel,i}=&\ r_{1,i}-r_{2,i}\\
  P_i=&\ p_{1,i}+p_{2,i}&p_{rel,i}=&\ \frac{m_2p_{1,i}-m_1p_{2,i}}{m_1+m_2}
\end{align*}
The corresponding masses and trapping frequencies are,
\begin{align*}
  M=&\ m_1+m_2&\mu=&\ \frac{m_1m_2}{m_1+m_2}\\
  \Omega_i^2=&\ \frac{m_1\omega_{1,i}^2+m_2\omega_{2,i}^2}{m_1+m_2}&\omega_{rel,i}^2=&\ \frac{m_2\omega_{1,i}^2+m_1\omega_{2,i}^2}{m_1+m_2}
\end{align*}
and the Hamiltonian can be expressed as,
\begin{align*}
  H=&\sum_{i=x,y,z}\paren{\frac{M\Omega_{i}^2R_{i}^2}{2}+\frac{P_{i}^2}{2M}}+
      \left[\sum_{i=x,y,z}\paren{\frac{\mu\omega_{rel,i}^2r_{rel,i}^2}{2}+\frac{p_{rel,i}^2}{2\mu}}+
      V_{int}\paren{\mathbf{r}_{rel}}\right]+
      \sum_{i=x,y,z}\mu\paren{\omega_{1,i}^2 - \omega_{2,i}^2}R_ir_{rel,i}
\end{align*}
The first term and the second term only relies on the COM motion and relative motion
respectively and can be solved independently. The third term mixes the COM and relative motion
and is proportional to the trapping frequency difference.
If the trapping frequencies are the same for the two atoms, the third term is $0$ and
the solution is fully separateable.
As mentioned above, since the trapping frequencies for the two atoms are similar,
we will assume the mixing term is small and treat it as a small correction in the calculation.

\subsection{Perturbative Calculation}

For weak interaction, i.e. a small scattering length $a$, the effect of the interaction
on the energy level can be calculated perturbatively.
The result from this calculation is useful for checking the validity of the full calculation,
as well as providing an intuitive understanding of the shift and its dependency
on different parameters.

For simplicity, we will assume all the trapping frequencies are the same,
i.e. $\omega_{1,i}=\omega_{2,i}=\omega_{rel,i}=\Omega_i=\omega_i$,
so that we only need to consider the relative motion,
\[
  H_{rel}=\sum_{i=x,y,z}\paren{\frac{\mu\omega_{i}^2r_{rel,i}^2}{2}+\frac{p_{rel,i}^2}{2\mu}}+
  V_{int}\paren{\mathbf{r}_{rel}}
\]
When treating the interaction as perturbation, the base solution is the harmonic oscillator
states for the relative motion $|n_{rel,x},n_{rel,y},n_{rel,z}\rangle$.
The energy level pertubation is then,
\begin{align*}
  \Delta_{n_{rel,x},n_{rel,y},n_{rel,z}}=&\langle n_{rel,x},n_{rel,y},n_{rel,z}|V_{int}\paren{\mathbf{r}_{rel}}|n_{rel,x},n_{rel,y},n_{rel,z}\rangle\\
  =&\frac{2\pi\hbar^2a}{\mu}\langle n_{rel,x},n_{rel,y},n_{rel,z}|\delta_{reg}\paren{\mathbf{r}_{rel}}|n_{rel,x},n_{rel,y},n_{rel,z}\rangle\\
  =&\frac{2\pi\hbar^2a}{\mu}\abs{\psi_{n_{rel,x},n_{rel,y},n_{rel,z}}(0)}^2\numberthis{eq:interaction-shift-perturb-shift}
\end{align*}
where $\abs{\psi_{n_{rel,x},n_{rel,y},n_{rel,z}}(0)}^2$ is the probability density
for zero distance between the atoms.

For the motional ground state, the shift is,
\begin{align*}
  \Delta_{0,0,0}=&a\frac{2\hbar^2}{\mu\sqrt{\pi}}\prod_{i=x,y,z}\frac{1}{\beta_{rel,i}}
\end{align*}
where $\beta_{rel,i}\equiv\sqrt{\hbar/\mu\omega_{rel,i}}$ is the relative motion
oscillator length along the $i$-th axis.
The shift is proportional to the strength of the interaction $a$,
and is also stronger for stronger confinement where the wavefunction density is higher.

We can see from (\ref{eq:interaction-shift-perturb-shift}) that the shift is only
non-zero when all of $n_{rel,i}$'s are even.
The shift is also smaller for higher motional excited state with smaller wavefunction density.
This means that the shift will only be observable if the atom is cooled to closed to
the motional ground state and will be small or zero for hot atoms.

\subsection{Non-perturbative Calculation}

(Exact solution for anisotripic trap)
(Correction on top of exact solution)

\section{Interaction Shift Spectroscopy}
(Sequence)
(Result)
(motional sideband, scattering length result)

\section{Summary and Outlook}
(Motional state selection)
