% -*- mode: latex-mode; TeX-engine: xetex; LaTeX-command-style: (("" "SOURCE_DATE_EPOCH=0 %(PDF)%(latex) --shell-escape %S%(PDFout)")); TeX-master: "../dissertation.tex"; -*-

\chapter{Two-photon Spectroscopy of NaCs Ground State}
\label{ch:raman-spectroscopy}

\section{Introduction}

\todo{make sure the goal of reaching the absolute ground state is stated previously.}

The excited molecular states measured and characterized in chapter \ref{ch:pa}
provide us a pathway to couple to the ground electronic molecular states
using two-photon transitions.
While it is in principle possible to drive from the atomic state
to any desired molecular ground state for various applications,
doing so directly has many technical challenges.
We will cover these challenges as well as the considerations in selecting
the molecule formation pathway in chapter \ref{ch:raman-transfer},
however, the difficulty, and the main difference from a pure atomic Raman transition,
lies in the wavefunction size mismatch.
As we have seen already in section \ref{pa:beampath},
the size mismatch between the excited molecular states and the ground atomic state
causes a very small FCF and requires a high PA intensity to improve the signal strength.
This small FCF also reduces the Rabi frequency for the two-photon transition to the ground state.
As a result, driving a two-photon transition to an arbitrary molecular ground state
may require maintaining coherence between two different lasers over
a relatively long time (milliseconds) which is very difficult to achieve.

Our solution to this challenge is to do the transfer via a two step process.
\begin{enumerate}
\item We drive a two-photon transition from the atomic state to
  a weakly bound ground state.
  The reduced energy difference allows the laser coherence to be maintained
  over a longer time easily.
  In the case of NaCs molecule, this also increases the FCF
  between the ground and excited molecular states which allows shorter pulse time and
  further reduces the coherence requirement.
\item The transfer to arbitrary molecular ground state will be done from
  the weakly bound state created in the first step.
  The strength of this transition can be much higher
  and only requires a relatively shorter laser coherence time.
  This step has already been demonstrated in other experiments\todo{\cite{}}.
\end{enumerate}

\todo{
  Setup/sequence/states
  N=0 bound states for different HF states
  N=2 bound states/field dependency
}

\section{Weakly bound NaCs Ground States}

\todo{
  Stable HF states
}
\todo{mention use of tweezer as Raman, referenced in next chapter}

\section{Raman Resonance on $v''=-1,\ N=0$ Ground State}

\section{Raman Resonance on $v''=-1,\ N=2$ Ground State}
\todo{
  (NaCs 2017-2018 > NaCs 2018 > Raman transfer 10/5 - 10/8 > v=-1 N=2 properties 11/03 - 11/07)
}
