% -*- mode: latex-mode; TeX-engine: xetex; LaTeX-command-style: (("" "SOURCE_DATE_EPOCH=0 %(PDF)%(latex) --shell-escape %S%(PDFout)")); TeX-master: "../dissertation.tex"; -*-

\chapter{Two-photon Spectroscopy of NaCs Ground State}
\label{ch:raman-spectroscopy}

\section{Introduction}

\todo{make sure the goal of reaching the absolute ground state is stated previously.}

The excited molecular states measured and characterized in chapter \ref{ch:pa}
provide us a pathway to couple to the ground electronic molecular states
using two-photon transitions.
While it is in principle possible to drive from the atomic state
to any desired molecular ground state for various applications,
doing so directly has many technical challenges.
We will cover these challenges as well as the considerations in selecting
the molecule formation pathway in chapter \ref{ch:raman-transfer},
however, the difficulty, and the main difference from a pure atomic Raman transition,
lies in the wavefunction size mismatch.
As we have seen already in section \ref{pa:beampath},
the size mismatch between the excited molecular states and the ground atomic state
causes a very small FCF and requires a high PA intensity to improve the signal strength.
This small FCF also reduces the Rabi frequency for the two-photon transition to the ground state.
As a result, driving a two-photon transition to an arbitrary molecular ground state
may require maintaining coherence between two different lasers over
a relatively long time (milliseconds) which is very difficult to achieve.

Our solution to this challenge is to do the transfer via a two step process.
\begin{enumerate}
\item We drive a two-photon transition from the atomic state to
  a weakly bound ground state.
  The reduced energy difference allows the laser coherence to be maintained
  over a longer time easily.
  In the case of NaCs molecule, this also increases the FCF
  between the ground and excited molecular states which allows shorter pulse time and
  further reduces the coherence requirement.
\item The transfer to arbitrary molecular ground state will be done from
  the weakly bound state created in the first step.
  The strength of this transition can be much higher
  and only requires a relatively shorter laser coherence time.
  This step has already been demonstrated in other experiments\todo{\cite{}}
  so in this thesis we will focus only on the first step transfer.
\end{enumerate}

In this chapter, we will discuss the use of Raman spectroscopy
to measure the properties of the weakly bound molecular ground states.
In section \ref{ch:raman-spectroscopy:states}
we will describe the states involved and the setup for the Raman spectroscopy
as well as the measured binding energy for the $N=0$ states.
In section \ref{ch:raman-spectroscopy:n2}
we study the coupling between angular momentums for near threshold molecular states
by characterizing the $N=2$ states.

\section{Weakly bound NaCs Ground States}
\label{ch:raman-spectroscopy:states}

As mentioned in section \ref{pa:structure:near-threshold},
the angular momentum coupling for weakly bound molecular state is similar to that of the atoms.
Therefore, instead of using the term symbol for the \textit{Hund's case (a)}
to identify the molecular potential and bound states,
we use the hyperfine state ($F_{Na}, F_{Cs}$) for the atoms instead.

\todo{figure?}
In order to measure the binding energy of a molecular state,
we first prepare the atom in the corresponding hyperfine state
and drive a Raman transition to the molecular state.
We use a Raman transition that is detuned from the $c^3\Sigma\ v'=0$ state measured
in section \label{pa:pa}.
\todo{Maybe move stable HF combination to interaction shift?}
When the Na and Cs are in the same tweezer,
they can undergo fast spin-exchange collision that changes the hyperfine state of the atom.
This process can cause the hyperfine energy ($>1\mathrm{GHz}$) of the atoms
to be transferred to the motional energy
and eject the atoms from the tweezer ($<100\mathrm{MHz}$ deep).
As a result, the measurement can only be done when the spin-exchange collision is suppressed,
which includes the following spin combinations,
\begin{enumerate}
\item $F_{Na}=1$ and $F_{Cs}=3$\\
  This is the spin state with the lowest energy and therefore the spin-exchange interaction
  is energetically forbidden.
  In the experiment, we use the state $|Na(1, 1),Cs(3, 3)\rangle$
  which can be prepared from the $|Na(2, 2),Cs(4, 4)\rangle$ state from OP
  easily by driving a Raman transition for both Na and Cs atoms.
  This state also remains the lowest energy atomic state in the present of a weak magnetic field.
\item $|Na(2, 2),Cs(4, 4)\rangle$ and $|Na(2, 2),Cs(3, 3)\rangle$
  \footnote{States with opposite $m_F$, i.e.
    $|Na(2, -2),Cs(4, -4)\rangle$ and $|Na(2, -2),Cs(3, -3)\rangle$ are also stable
    but are omitted here since these cannot be easily prepared in our experiment.}\\
  These spin states are stable because the spin-exchange collision conserves total $m_F$
  of the two atoms and
  the two states are the lowest energy states that has the same total $m_F$.
  Inelastic collision that changes the total $m_F$ can also happen
  but has a lower collision rate since it requires transferring angular momentum
  between the spin and motion of the atom.
\end{enumerate}

\subsection{Driving Raman Transition using the Optical Tweezer}

\todo{
  Setup/sequence
  mention use of tweezer as Raman, referenced in next chapter
}

\subsection{Raman Resonance on $v''=-1,\ N=0$ Ground State}
\todo{
  N=0 bound states for different HF states
  B field dependency?
}

\section{Angular momentum Coupling in $N=2$ Ground State}
\label{ch:raman-spectroscopy:n2}

\todo{
  N=2 bound states/field dependency
}
\todo{
  (NaCs 2017-2018 > NaCs 2018 > Raman transfer 10/5 - 10/8 > v=-1 N=2 properties 11/03 - 11/07)
}
