% -*- mode: latex-mode; TeX-engine: xetex; LaTeX-command-style: (("" "SOURCE_DATE_EPOCH=0 %(PDF)%(latex) --shell-escape %S%(PDFout)")); TeX-master: "../dissertation.tex"; -*-

\chapter{Computer Control Hardware Specification}
\label{appendex:computer-control}

In this appendex we list the specification of the important hardware
used in our computer control system.
See section~\ref{ch:computer-control:backend} for the integration
of these hardware into the system.

\section{FPGA}
\label{appendex:computer-control:fpga}

We use the \href{https://www.xilinx.com/products/boards-and-kits/ek-z7-zc702-g.html}{ZC702 evaluation board}~(part number EK-Z7-ZC702-G).
The on board CPU has a maximum clock speed of $666.667~\mathrm{MHz}$
and supports the VFPv3 and NEON extension for floating point and SIMD instructions.
The board also includes $1~\mathrm{GiB}$ of DDR3 RAM connected to the CPU.
The FPGA is configured to run at $100~\mathrm{MHz}$ which determines
the highest timing resolution of $10~\mathrm{ns}$ in our experiment.

We connect the FPGA to the peripherals using two FMC LPC connectors
each containing 68 pins used for single-ended signals.
Each FMC connector is used to control 11 DDS's,
which will be described in section~\ref{appendex:computer-control:fpga},
and one of the connector is also used to output 32 logical control signals
and the clock to synchronize with other devices.

\todo{
  DDS connection
  potential concurrency capability
}

\section{DDS}
\label{appendex:computer-control:fpga}

\todo{
  Part number
  Clock speed, bandwidth
  output amplifier, maximum power, noise, cross talk minimized by shielding
  Programming mode
}

\section{NI DAQ Card}

\todo{
  Part number
  Voltage range, resolution
  refresh speed
  external trigger and clock
  Ground isolation
}

\section{USRP}

\todo{
  Part number
  Daughter board, noise with higher frequency daughter board
  sampling rate/bandwidth
  output power
}
