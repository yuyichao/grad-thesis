% -*- mode: latex; TeX-engine: xetex; LaTeX-command-style: (("" "SOURCE_DATE_EPOCH=0 %(PDF)%(latex) --shell-escape %S%(PDFout)")); TeX-master: "../dissertation.tex"; -*-

%% the acknowledgments section
Many thanks are owed to Kang-Kuen Ni, my advisor,
for the vision on the experiment and giving me the opportunity
to participate in extending the AMO toolbox and our understanding of quantum physics
as a graduate student.
I also learnt from her the way to do research and to manage an experiment,
especially on focusing on the big picture and setting the correct priority.

Of course, none of the work included in this thesis would have been possible
without the collaboration and support from many other people in the lab.
I would like to thank the following people who have worked on our apparatus:
Lee Liu, Nick Hutzler, Till Rosenband, Jessie Zhang,
Jonathan Hood, Kenneth Wang and Lewis Picard,
as well as all the members of the two other experiments in our lab
for their constant support and sharing of equipment.
I would like to acknowledge in particular:
Lee Liu, our first graduate student~(now postdoc at JILA),
who was in charge of the building of our experiment and almost single-handedly implemented
every preparation steps we have on the Cs atoms;
Nick Hutzler, our first postdoc~(now Assistant Professor at Caltech),
who helped with a lot of the early building work using his abundant experience
as experimental physicists and solved the loading problem for Na atoms
with a brilliant method~(see section~\ref{ch:loading:loading});
Till Rosenband, for providing us with reliably devices
and for designing and implementing the first prototype
of the computer control system~(see section~\ref{ch:computer-control});
Jonathan Hood, our second postdoc~(now Assistant Professor at Purdue),
who joint our effort on making molecules after Nick left and, among other things,
taught me everything I know now about diatomic molecules;
and Kenneth Wang, our current graduate student,
who is now leading the experiment to a new level.

Being part of the Harvard-MIT Center for Ultracold Atoms~(CUA) is a blessing to our experiment.
We were able to share ideas, techniques and hardware with other groups within the CUA
and we got many inspirations for the problems we faced during our discussions.
Among them, I would like to personally thank the members of my undergraduate experiment BEC-5,
Jesse Amato-Grill, Ivana Dimitrova, Niklas Jepsen and Will Lundon
and the advisor Wolfgang Ketterle, who introduced me to the field of AMO
and taught me lessons that were invaluable throughout my PhD.

Lastly, my mom, Haiying Yan, and my dad, Xi Yu, are my great teachers
and introduced me to the wonder world of science since I was a kid.
They have also been supporting my decisions during my time in the U.S..
I would also like to thank my cousin, Cheng Qian,
whom I have been following the footsteps of in schools.
As my only close relative in the U.S., she supported me from New York City
when my parents were not able to from across the sea.
