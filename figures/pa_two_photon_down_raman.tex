\documentclass[border=3pt]{standalone}
\usepackage{amsmath}
\usepackage{amssymb}
\usepackage{amsfonts}
\usepackage{esint}
\usepackage{bbm}
\usepackage{amscd}
\usepackage{caption}
\usepackage{picinpar}
\usepackage{graphicx}
\usepackage{tikz}
\usepackage{tikz-3dplot}
\usepackage{indentfirst}
\usepackage{wrapfig}
\usepackage{units}
\usepackage{textcomp}
\usepackage[utf8x]{inputenc}
% \usepackage{feyn}
\usepackage{feynmp}
\usepackage{xkeyval}
\usepackage{xargs}
\usepackage{verbatim}
\usepackage{pgfplots}
\usepackage{hyperref}
\usetikzlibrary{
  arrows,
  calc,
  decorations.pathmorphing,
  decorations.pathreplacing,
  decorations.markings,
  positioning,
  shapes
}

\DeclareGraphicsRule{*}{mps}{*}{}
\newcommand{\ud}{\mathrm{d}}
\newcommand{\ue}{\mathrm{e}}
\newcommand{\ui}{\mathrm{i}}
\newcommand{\res}{\mathrm{Res}}
\newcommand{\Tr}{\mathrm{Tr}}
\newcommand{\dsum}{\displaystyle\sum}
\newcommand{\dprod}{\displaystyle\prod}
\newcommand{\dlim}{\displaystyle\lim}
\newcommand{\dint}{\displaystyle\int}
\newcommand{\fsno}[1]{{\!\not\!{#1}}}
\newcommand{\eqar}[1]
{
  \begin{align*}
    #1
  \end{align*}
}
\newcommand{\texp}[2]{\ensuremath{{#1}\times10^{#2}}}
\newcommand{\dexp}[2]{\ensuremath{{#1}\cdot10^{#2}}}
\newcommand{\eval}[2]{{\left.{#1}\right|_{#2}}}
\newcommand{\paren}[1]{{\left({#1}\right)}}
\newcommand{\lparen}[1]{{\left({#1}\right.}}
\newcommand{\rparen}[1]{{\left.{#1}\right)}}
\newcommand{\abs}[1]{{\left|{#1}\right|}}
\newcommand{\sqr}[1]{{\left[{#1}\right]}}
\newcommand{\crly}[1]{{\left\{{#1}\right\}}}
\newcommand{\angl}[1]{{\left\langle{#1}\right\rangle}}
\newcommand{\tpdiff}[4][{}]{{\paren{\frac{\partial^{#1} {#2}}{\partial {#3}{}^{#1}}}_{#4}}}
\newcommand{\tpsdiff}[4][{}]{{\paren{\frac{\partial^{#1}}{\partial {#3}{}^{#1}}{#2}}_{#4}}}
\newcommand{\pdiff}[3][{}]{{\frac{\partial^{#1} {#2}}{\partial {#3}{}^{#1}}}}
\newcommand{\diff}[3][{}]{{\frac{\ud^{#1} {#2}}{\ud {#3}{}^{#1}}}}
\newcommand{\psdiff}[3][{}]{{\frac{\partial^{#1}}{\partial {#3}{}^{#1}} {#2}}}
\newcommand{\sdiff}[3][{}]{{\frac{\ud^{#1}}{\ud {#3}{}^{#1}} {#2}}}
\newcommand{\tpddiff}[4][{}]{{\left(\dfrac{\partial^{#1} {#2}}{\partial {#3}{}^{#1}}\right)_{#4}}}
\newcommand{\tpsddiff}[4][{}]{{\paren{\dfrac{\partial^{#1}}{\partial {#3}{}^{#1}}{#2}}_{#4}}}
\newcommand{\pddiff}[3][{}]{{\dfrac{\partial^{#1} {#2}}{\partial {#3}{}^{#1}}}}
\newcommand{\ddiff}[3][{}]{{\dfrac{\ud^{#1} {#2}}{\ud {#3}{}^{#1}}}}
\newcommand{\psddiff}[3][{}]{{\frac{\partial^{#1}}{\partial{}^{#1} {#3}} {#2}}}
\newcommand{\sddiff}[3][{}]{{\frac{\ud^{#1}}{\ud {#3}{}^{#1}} {#2}}}
\usepackage{fancyhdr}
\usepackage{multirow}
\usepackage{fontenc}
% \usepackage{tipa}
\usepackage{ulem}
\usepackage{color}
\usepackage{cancel}
\newcommand{\hcancel}[2][black]{\setbox0=\hbox{#2}%
  \rlap{\raisebox{.45\ht0}{\textcolor{#1}{\rule{\wd0}{1pt}}}}#2}

\newcommand\pgfmathsinandcos[3]{%
  \pgfmathsetmacro#1{sin(#3)}%
  \pgfmathsetmacro#2{cos(#3)}%
}
\newcommand\LongitudePlane[3][current plane]{%
  \pgfmathsinandcos\sinEl\cosEl{#2} % elevation
  \pgfmathsinandcos\sint\cost{#3} % azimuth
  \tikzset{#1/.estyle={cm={\cost,\sint*\sinEl,0,\cosEl,(0,0)}}}
}
\newcommand\LatitudePlane[3][current plane]{%
  \pgfmathsinandcos\sinEl\cosEl{#2} % elevation
  \pgfmathsinandcos\sint\cost{#3} % latitude
  \pgfmathsetmacro\yshift{\cosEl*\sint}
  \tikzset{#1/.estyle={cm={\cost,0,0,\cost*\sinEl,(0,\yshift)}}} %
}
\newcommand\DrawLongitudeCircle[2][1]{
  \LongitudePlane{\angEl}{#2}
  \tikzset{current plane/.prefix style={scale=#1}}
  % angle of "visibility"
  \pgfmathsetmacro\angVis{atan(sin(#2)*cos(\angEl)/sin(\angEl))} %
  \draw[current plane] (\angVis:1) arc (\angVis:\angVis+180:1);
  \draw[current plane,dashed] (\angVis-180:1) arc (\angVis-180:\angVis:1);
}
\newcommand\DrawLatitudeCircleArrow[2][1]{
  \LatitudePlane{\angEl}{#2}
  \tikzset{current plane/.prefix style={scale=#1}}
  \pgfmathsetmacro\sinVis{sin(#2)/cos(#2)*sin(\angEl)/cos(\angEl)}
  % angle of "visibility"
  \pgfmathsetmacro\angVis{asin(min(1,max(\sinVis,-1)))}
  \draw[current plane,decoration={markings, mark=at position 0.6 with {\arrow{<}}},postaction={decorate},line width=.6mm] (\angVis:1) arc (\angVis:-\angVis-180:1);
  \draw[current plane,dashed,line width=.6mm] (180-\angVis:1) arc (180-\angVis:\angVis:1);
}
\newcommand\DrawLatitudeCircle[2][1]{
  \LatitudePlane{\angEl}{#2}
  \tikzset{current plane/.prefix style={scale=#1}}
  \pgfmathsetmacro\sinVis{sin(#2)/cos(#2)*sin(\angEl)/cos(\angEl)}
  % angle of "visibility"
  \pgfmathsetmacro\angVis{asin(min(1,max(\sinVis,-1)))}
  \draw[current plane] (\angVis:1) arc (\angVis:-\angVis-180:1);
  \draw[current plane,dashed] (180-\angVis:1) arc (180-\angVis:\angVis:1);
}
\newcommand\coil[1]{
  {\rh * cos(\t * pi r)}, {\apart * (2 * #1 + \t) + \rv * sin(\t * pi r)}
}
\makeatletter
\define@key{DrawFromCenter}{style}[{->}]{
  \tikzset{DrawFromCenterPlane/.style={#1}}
}
\define@key{DrawFromCenter}{r}[1]{
  \def\@R{#1}
}
\define@key{DrawFromCenter}{center}[(0, 0)]{
  \def\@Center{#1}
}
\define@key{DrawFromCenter}{theta}[0]{
  \def\@Theta{#1}
}
\define@key{DrawFromCenter}{phi}[0]{
  \def\@Phi{#1}
}
\presetkeys{DrawFromCenter}{style, r, center, theta, phi}{}
\newcommand*\DrawFromCenter[1][]{
  \setkeys{DrawFromCenter}{#1}{
    \pgfmathsinandcos\sint\cost{\@Theta}
    \pgfmathsinandcos\sinp\cosp{\@Phi}
    \pgfmathsinandcos\sinA\cosA{\angEl}
    \pgfmathsetmacro\DX{\@R*\cost*\cosp}
    \pgfmathsetmacro\DY{\@R*(\cost*\sinp*\sinA+\sint*\cosA)}
    \draw[DrawFromCenterPlane] \@Center -- ++(\DX, \DY);
  }
}
\newcommand*\DrawFromCenterText[2][]{
  \setkeys{DrawFromCenter}{#1}{
    \pgfmathsinandcos\sint\cost{\@Theta}
    \pgfmathsinandcos\sinp\cosp{\@Phi}
    \pgfmathsinandcos\sinA\cosA{\angEl}
    \pgfmathsetmacro\DX{\@R*\cost*\cosp}
    \pgfmathsetmacro\DY{\@R*(\cost*\sinp*\sinA+\sint*\cosA)}
    \draw[DrawFromCenterPlane] \@Center -- ++(\DX, \DY) node {#2};
  }
}
\makeatother
\tikzstyle{snakearrow} = [decorate, decoration={pre length=0.2cm,
  post length=0.2cm, snake, amplitude=.4mm,
  segment length=2mm},thick, ->]
%% document-wide tikz options and styles
\tikzset{%
  >=stealth, % option for nice arrows
  inner sep=0pt,%
  outer sep=2pt,%
  mark coordinate/.style={inner sep=0pt,outer sep=0pt,minimum size=3pt,
    fill=black,circle}%
}

\makeatletter
\long\def\my@drawfill#1#2;{%
  \@skipfalse
  \fill[#1,draw=none] #2;
  \@skiptrue
  \draw[#1,fill=none] #2;
}
\newif\if@skip
\newcommand{\skipit}[1]{%
  \if@skip
  \else
  #1
  \fi
}
\newcommand{\drawfill}[1][]{%
  \my@drawfill{#1}}
\makeatother

\ifpdf
% Ensure reproducible output
\pdfinfoomitdate=1
\pdfsuppressptexinfo=-1
\pdftrailerid{}
\hypersetup{
  pdfcreator={},
  pdfproducer={}
}
\fi

\begin{document}

\begin{tikzpicture}
  \draw[->, line width=2] (0, -2.5) -- node[rotate=90,above] {\textbf{Energy}} (0, 4.9);
  \begin{scope}[shift={(0.3, 0)}]
    \draw[line width=1] (0.5, -0.8) node[left] {$|m\rangle$} -- (9.4, -0.8);
    \draw[line width=1] (0.5, 0) node[left] {$|g\rangle$} -- (9.4, 0);
    \draw (0.5, 0.1) -- (9.4, 0.1);
    \draw (0.5, 0.6) -- (9.4, 0.6);
    \fill[opacity=0.2] (0.5, 0.7) rectangle (9.4, 1.2);

    % \draw[densely dotted,line width=0.7] (9.4, 0.7) -- (9.6, 0.7);
    % \draw[line width=1] (0.5, 3.5) node[left] {$|e\rangle$} -- (9.4, 3.5);
    % \draw[decoration={brace,mirror,raise=1pt, amplitude=3pt},decorate,line width=0.8]
    % (9.45, 0) -- node[right=4pt] {\scriptsize Trap States} (9.45, 0.67);
    % \draw[->,line width=0.8] (9.5, 0.7) --
    % node[right] {\scriptsize Continuum} (9.5, 1.4);

    \begin{scope}[shift={(0, 0)}]
      \node at (1.0, 4.2) {(\textbf{A})};
      \draw[dashed] (0.5, 3.0) -- (1.7, 3.0);
      \node[rotate=90] at (1.1, 0.4) {\small $\cdots$};
      \draw[->, line width=1,green!80!black] (1.11, 3.0) --
      node[right,pos=0.23684] {$\omega_2$} (1.5, -0.8);
      \draw[->, line width=1,blue] (0.7, 0) --
      node[left,pos=0.7] {$\omega_1$} (1.09, 3.0);
    \end{scope}

    \begin{scope}[shift={(1.5, 0)}]
      \node at (1.0, 4.2) {(\textbf{B})};
      \node[rotate=90] at (1.1, 0.4) {\small $\cdots$};
      \draw[<-, line width=1,blue] (1.5, 0) --
      node[right,pos=0.7] {$\omega_1$} (1.11, 3.0);
      \draw[<-, line width=1,green!80!black] (1.09, 3.0) --
      node[left,pos=0.23684] {$\omega_2$} (0.7, -0.8);
    \end{scope}

    \begin{scope}[shift={(3.0, 0)}]
      \node at (1.0, 4.2) {(\textbf{C})};
      \node[rotate=90] at (1.1, 0.4) {\small $\cdots$};
      \draw[->, line width=1,green!80!black] (1.11, 3.0) --
      node[right,pos=0.23684] {$\omega_2$} (1.5, -0.8);
      \draw[<-, line width=1,green!80!black] (1.09, 3.0) --
      node[left,pos=0.23684] {$\omega_2$} (0.7, -0.8);
    \end{scope}

    \begin{scope}[shift={(4.5, 0)}]
      \node at (1.0, 4.2) {(\textbf{D})};
      \node[rotate=90] at (1.1, 0.4) {\small $\cdots$};
      \draw[->, line width=1,blue] (1.11, 2.2) --
      node[right,pos=0.18] {$\omega_1$} (1.5, -0.8);
      \draw[<-, line width=1,blue] (1.09, 2.2) --
      node[left,pos=0.18] {$\omega_1$} (0.7, -0.8);
    \end{scope}

    \begin{scope}[shift={(6.0, 0)}]
      \node at (1.0, 4.2) {(\textbf{E})};
      \node[rotate=90] at (1.1, 0.4) {\small $\cdots$};
      \draw[<-, line width=1,blue] (1.09, 2.2) --
      node[left,pos=0.18] {$\omega_1$} (0.7, -0.8);
      \draw[->, line width=1,green!80!black] (1.11, 2.2) --
      node[right,pos=0.14211] {$\omega_2$} (1.5, -1.6);
    \end{scope}

    \begin{scope}[shift={(7.5, 0)}]
      \node at (1.0, 4.2) {(\textbf{F})};
      \fill[red,opacity=0.2] (1.15, -1.7) rectangle (1.75, 1.2);
      \node[red,rotate=90] at (1.45, -1.9) {\small $\cdots$};
      \draw[red] (1.15, -2.2) -- (1.75, -2.2);
      \draw[red, line width=1] (1.15, -2.3) node[left] {$|g'\rangle$} -- (1.75, -2.3);
      \node[rotate=90] at (1.1, 0.4) {\small $\cdots$};
      \draw[<-, line width=1,blue] (1.09, 2.2) --
      node[left,pos=0.18] {$\omega_1$} (0.7, -0.8);
      \draw[->, line width=1,green!80!black] (1.11, 2.2) --
      node[right,pos=0.14211] {$\omega_2$} (1.5, -1.6);
    \end{scope}
  \end{scope}
\end{tikzpicture}

\end{document}
