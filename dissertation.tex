% -*- mode: latex-mode; TeX-engine: xetex; LaTeX-command-style: (("" "SOURCE_DATE_EPOCH=0 %(PDF)%(latex) --shell-escape %S%(PDFout)")); -*-

\documentclass{Dissertate}

\usepackage{fltpage2}
\usepackage{multirow, makecell}
\usepackage{tabularx}
\usepackage[perpage]{footmisc}

\newcolumntype{Y}{>{\centering\arraybackslash}X}

\newcommand{\todo}[1]{}
\newcommand{\ud}{\mathrm{d}}
\newcommand{\ue}{\mathrm{e}}
\newcommand{\ui}{\mathrm{i}}
\newcommand{\Na}{\mathrm{Na}}
\newcommand{\Cs}{\mathrm{Cs}}
\newcommand{\abs}[1]{{\left|{#1}\right|}}
\newcommand{\paren}[1]{{\left({#1}\right)}}
\newcommand{\numberthis}[1]{\addtocounter{equation}{1}\tag{\theequation}\label{#1}}

\begin{document}

% Formatting guidelines found in:
% http://www.gsas.harvard.edu/publications/form_of_the_phd_dissertation.php
% -*- mode: latex-mode; TeX-engine: xetex; LaTeX-command-style: (("" "SOURCE_DATE_EPOCH=0 %(PDF)%(latex) --shell-escape %S%(PDFout)")); TeX-master: "../dissertation.tex"; -*-

% Some details about the dissertation.
\title{Coherent Creation of Single Molecules from Single Atoms}
\author{Yichao Yu}

%If you have one advisor
\advisor{Kang-Kuen Ni}

% \committeeInternalOne{Person Inside One}
% \committeeInternalTwo{Person Inside Two}
% \committeeExternal{Person Outside}

% ... about the degree.
\degree{Doctor of Philosophy}
\field{Physics}
\degreeyear{2021}
\degreeterm{Winter}
\degreemonth{March}
\department{Physics}

% ... about the candidate's previous degrees.
\pdOneName{B.S.}
\pdOneSchool{Massachusetts Institute of Technology}
\pdOneYear{2014}

\makeatletter
\hypersetup{
  pdfauthor={\@author},
  pdftitle={\@title},
  pdfsubject={Ph.D Thesis},
  pdfkeywords={Physics, AMO, Laser, Atomic Physics, Molecular Physics, Optics,
  Ultracold, Ultracold Atoms, Ultracold Molecules, Diatomic Molecules,
  Sodium, Cesium, NaCs, Coherent Transfer, Single Atoms, Single Molecule, Optical Tweezer,
  Molecule Assembly, Raman Transition, STIRAP, Raman Sideband Cooling,
  Photoassociation, Computer Control, LLVM, FPGA}
}
\makeatother

\maketitle
\copyrightpage
\abstractpage{% -*- mode: latex-mode; TeX-engine: xetex; LaTeX-command-style: (("" "SOURCE_DATE_EPOCH=0 %(PDF)%(latex) --shell-escape %S%(PDFout)")); TeX-master: "../dissertation.tex"; -*-

%% the abstract
\todo{
  neutral Molecules have structures/dipole interactions.
  Molecules are promising for ...
  In order to take full advantage, need high level of quantum control on single molecules.
  which poses high level of challenge due to
  multiple approaches has been taken like .... which requires ... or ....

  In this thesis, we successfully achieved full quantum control
  on a stable weakly bound molecule associated via optical transfer from atoms.
  use optical tweezer to achieve control
  optical tweezers also allows scaling up.
  Studied various interaction and molecule properties.
}
}
\contentspage
% \listoffigures % optional
% \dedicationpage{\input{frontmatter/dedication}}
\acknowledgments{% -*- mode: latex; TeX-engine: xetex; LaTeX-command-style: (("" "SOURCE_DATE_EPOCH=0 %(PDF)%(latex) --shell-escape %S%(PDFout)")); TeX-master: "../dissertation.tex"; -*-

%% the acknowledgments section

\todo{
  Advisor
  Lab members + Till
  CUA
  Undergrad lab members and advisor
  Family and friends
}
}
\setstretch{\dnormalspacing}

% include each chapter...
\setcounter{chapter}{-1}  % start chapter numbering at 0
% -*- mode: latex; TeX-engine: xetex; LaTeX-command-style: (("" "SOURCE_DATE_EPOCH=0 %(PDF)%(latex) --shell-escape %S%(PDFout)")); TeX-master: "../dissertation.tex"; -*-

\chapter{Introduction}
\label{ch:introduction}

\section{Ultracold Molecules}
\label{ch:introduction:molecules}

Started with precise optical spectroscopy and the development of laser cooling
and trapping technologies, the increasing measurement precision and
high level of control has been one of the the primary driving forces
in the field of atomic physics in the past decades.
Systems based on cooling and controlling of neutral atoms
have been used in a wide range of applications including
quantum computing~\cite{wang_single-qubit_2016,anderlini_controlled_2007,
  kaufman_entangling_2015,levine_high-fidelity_2018,isenhower_demonstration_2010},
quantum communication~\cite{tiecke_nanophotonic_2014,reiserer_quantum_2014,
  welte_photon-mediated_2018},
precision measurement~\cite{bloom_optical_2014,parker_measurement_2018},
quantum simulation~\cite{bakr_quantum_2009,cheuk_quantum-gas_2015,bernien_probing_2017,
  leseleuc_observation_2019,koepsell_imaging_2019,chiu_string_2019,
  bloch_many-body_2008,lahaye_physics_2009}
and the study of many other many body effects~\cite{weiner_experiments_1999,
  sompet_thermally_2019,xu_interaction-induced_2015,guan_density_2019,
  dimitrova_observation_2017}.

Despite these great successes, the utility of atom based systems
are limited by their simple internal structure and weak interactions.
Molecules, on the other hand, with its richer internal degrees of freedom
including electronic orbital, electron spin, neuclear vibration, rotation and spin,
the potential of stronger interaction and more variance of symmetry classes,
are better candidates for more applications especially in
precision measurement~\cite{bickman_preparation_2009,hudson_improved_2011,eckel_search_2013,
  kozyryev_precision_2017,cairncross_precision_2017,acme_collaboration_improved_2018,
  kondov_molecular_2019,flambaum_electric_2020},
quantum simulations~\cite{micheli_toolbox_2006,buchler_strongly_2007,gorshkov_tunable_2011,
  baranov_condensed_2012,yao_realizing_2013,syzranov_spin-orbital_2014,
  wall_quantum_2015,wall_realizing_2015,peter_topological_2015,
  yao_quantum_2018,sundar_synthetic_2018,schuster_realizing_2019},
quantum information processing~\cite{demille_quantum_2002,ni_dipolar_2018,
  hudson_dipolar_2018,lin_quantum_2020},
and quantum chemistry~\cite{balakrishnan_perspective_2016,hu_direct_2019,
  segev_collisions_2019,jongh_imaging_2020}.
Moreover, comparing to other systems with stronger interaction like ions and Rydberg atoms,
the interacting states in molecules are long lived~\cite{fedorov_accurate_2014}
and tunable~\cite{giovanazzi_tuning_2002} thanks to the abundance of low energy excitations,
which can offer better isolation from the environment giving rise to longer coherence time.

Many applications of molecules requires cooling and high level of control
on the quantum state of the molecules.
Unfortunately, the properties that makes molecules attractive
also makes controlling them harder.
The enabling technique for most ultracold atom experiment, laser cooling,
typically requires scattering of a large number of photons ($\approx\!10^4\sim10^8$).
This is possible in atomic system due to the existence of cycling transition
or near cycling ones that can be completely closed with
one or two spin states repumping lasers.
However, the lack of selection rules for vibrational states means that
such transition generally does not exist in molecules.
As a result, experiments aiming to achieve control of molecule on a similar level with atoms
usually take one of the two approaches,
\begin{enumerate}
\item Direct laser cooling of special molecules
  with approximate cycling transitions~\cite{barry_magneto-optical_2014,
    norrgard_submillikelvin_2016,truppe_molecules_2017,
    anderegg_laser_2018,mccarron_magnetic_2018,collopy_3d_2018,
    mitra_direct_2020,ding_sub-doppler_2020}.\\
  These are molecules that have optical transitions that has a high probability
  of decaying down to the same vibrational states.
  With the help of a few vibrational repumping lasers,
  scattering of $\approx\!10^3\sim10^7$ photons can be achieved.
  Significant progress has been made using this approach in recent years
  including sub-doppler cooling~\cite{cheuk_mathrmlambda-enhanced_2018}
  and trapping of molecules in optical tweezers~\cite{anderegg_optical_2019}.
  The main challenge with this approach is to achieve a better cooling performance
  given the still limitted photon scattering budget.
\item Creation of molecules from ultracold atoms.\\
  First realized a decade ago~\cite{ni_high_2008,lang_ultracold_2008},
  this approach takes advantage of the mature cooling and trapping techniques
  developped for atoms and creates molecules from atoms
  that are cooled to ultracold temperature.
  The difficulty with this approach is understandably in the creation of the molecule,
  which must be done with acceptable efficiency and coherence
  in order to maintain the cooling and controling done on the atoms.
  This approach currently allows a lower temperature to be achieved~\cite{
    marco_degenerate_2019,zhang_forming_2020,he_coherently_2020}
  and this is the approach used in our experiment.
\end{enumerate}

\section{Assembly of Molecules in Optical Tweezers}
\label{ch:introduction:tweezers}

Previous experiments that takes the approach to create ultracold molecules
from ultracold atoms do so either in a bulk gases
or an optical lattice~\cite{moses_creation_2015}.
In these systems, however, the transfer efficiency and
the density of the created molecules are limited
by the overlap between the distribution of the two atomic speices.
Such overlap is controlled by the combination of the trapping potential
and the intra- and inter-species interactions.
Thus, a perfect overlap for molecule formation can only be achieved by fine tuning
the delicate balance between these parameters and may not always be possible.
Additionally, the molecules created in such matter may collide
with residue atoms or other molecules causing rapid loss
either through chemical reactions or by forming long-lived sticky complex~\cite{
  mayle_scattering_2013,croft_long-lived_2014,liu_photo-excitation_2020}.

Since these issues are essentially all caused by the lack of direct control
on the position and motion of individual atoms and molecules,
we propose a general solution using optical tweezers.
Created by focusing the trap light through a high numerical aperture~(NA) objective,
optical tweezers can trap single atoms in flexible geometry
and the setup naturally provides the high resolution required
for detection and manipulation of individual atoms~\cite{schlosser_sub-poissonian_2001}.
Taking advantage of these properties, full quantum control on atoms has been demostrated
by cooling and rearranging the tweezer based on the loading result~\cite{
  barredo_atom-by-atom_2016,endres_atom-by-atom_2016}.
This gives us a good starting point to deterministically create molecules
by directly merging pairs of atoms instead of
relying on stochastic process or the fine tuning of parameters in previous experiments.
The molecules created using this approach are also well isolated from each other,
preventing loss due to collisions.

\subsection{Two-Step All Optical Creation of Molecules}
\label{ch:introduction:tweezers:two-step}

In order to create the rovibronic ground state molecules from atoms,
the $\approx\!100~\mathrm{THz}$ binding energy must be removed coherently from the system,
which is may be done using a two-photon optical transition.
However, due to the significant mismatch in the wavefunction size
between the atomic motional ground state~($\approx\!1000\text{\AA}$)
and the rovibronic ground state of the molecule~($\approx\!4\text{\AA}$),
it is challenging to achieve a high Rabi frequency or short transition time,
causing technical difficulties in maintaining the laser coherence during the transition.
Because of this, the transfer to rovibronic ground state is typically done in two steps~\cite{
  danzl_quantum_2008,ni_high_2008,lang_ultracold_2008,takekoshi_ultracold_2014,
  molony_creation_2014,park_ultracold_2015,guo_creation_2016,rvachov_long-lived_2017,
  kondov_molecular_2019,voges_ultracold_2020}.
The atoms are first associated to a weakly bound molecule
where the coherence is easier to achieve due to the smaller energy difference.
After that the molecule can be driven to the rovibronic ground state more quickly,
relaxing the coherence requirement. See section~\ref{ch:raman-transfer:state-selction}
for a more detailed and systematic description of the challenges
and the selection of the transfer pathway.

So far, most experiments implemented the first step
by magnetoassociation using a magnetic Feshbach scattering resonance~\cite{
  ni_high_2008,zhang_forming_2020}.
The only exceptions are $\mathrm{Sr}_2$,
where narrow-linewidth ($\sim 20~\mathrm{kHz}$) excited states
are available and optical association can be driven coherently~\cite{
  reinaudi_optical_2012,stellmer_creation_2012}
and $^{87}\mathrm{Rb}^{85}\mathrm{Rb}$,
where there are molecular states bound by $1\sim2~\mathrm{MHz}$~\cite{he_coherently_2020}.
With these requirements, molecules involving non-magnetic atoms
or atoms without narrow intercombination lines remain difficult to associate.

In our experiment, we propose a different method using only optical transitions
and does not rely on an narrow excited state linewidth.
This is enabled by the confinement of the optical tweezer and
careful selection of the transition pathway,
which will be covered in more detail in later chapters.
We believe the approach we demostrate in our experiment is more general
than the previous ones,
and can be applied to most other molecules created from laser-coolable atoms.

\subsection{Experiment Plan}
\label{ch:introduction:tweezers:plan}

\begin{figure}
  \centering
  \includegraphics[width=\textwidth]{figures/introduction_steps.pdf}
  \caption[Molecule creation steps.]{
    Steps to create single rovibronic molecules in optical tweezers.
    \label{fig:introduction:tweezers:plan:steps}}
\end{figure}

The molecule we use to implement this approach is NaCs.
We selected a bialkali molecule in order to take advantage of
the wide range of existing techniques developped to cool and manipulate alkali atoms.
The molecule is also predicted to have a large
molecular fixed-frame dipole moments~($4.6~\mathrm{Debye}$)~\cite{
  dagdigian_molecular_1972,deiglmayr_calculations_2008}
in the singlet rovibronic ground state,
making it a good candidate to demostrate interaction and entanglement
after created in the tweezers.

The steps we propose to create the molecules in the tweezer are show in
Fig.~\ref{fig:introduction:tweezers:plan:steps} and are listed as follows,
\begin{enumerate}
\item Loading and cooling of single atoms in the tweezers.\\
  This is the step that give us the full quantum control on the atoms
  which will be translated to the control on the molecules later.
  Image of the atoms taken in this step can be used to rearrange the tweezers
  to acheive high filling fraction.
\item Merging of the atoms into a single tweezer.\\
  Using the precise control of the atoms from optical tweezers,
  we can merge a single Na and Cs that is trapped and cooled in separate tweezers
  into a single one deterministically.
\item Creation of weakly bound molecule.\\
  The pair of atoms in a single tweezer will be coherently associated
  to a weakly bound molecule using a Raman transition detuned from an excited molecular state.
  This is the critical step that transfers the control we achieved
  on the atoms to molecules and will be the main result of this thesis.
\item Creation of rovibronic ground state molecule.\\
  Finally, the weakly bound molecule will be driven to the rovibronic ground state
  using another two-photon transition~\cite{bergmann_coherent_1998}.
  This prepare the molecule in a state with strong dipole interaction
  and can be used for many previously mentioned applications.
\end{enumerate}

\section{Contents of this Thesis}
\label{ch:introduction:contents}

In this thesis, we describe the method we use to create a weakly bound ground state molecule
in the optical tweezer and the results leading up to it including the control of atoms
and the measurement on the interaction between the atoms and the molecular potential.

We start with chapter \ref{ch:computer-control} by giving a high level description of
the custom computer control system we use.
This system controls the timing of all the hardware outputs in the experiment
and is used to perform all the measurements in the rest of this thesis.

Chapter \ref{ch:loading} discusses the loading of the single atoms into the optical tweezer.
The discussion includes the preparation steps needed before the loading,
e.g. freespace cooling of atoms,
and the imaging of the atom in the tweezer,
which is the primary detection method used in our experiment.
All the experiments in the following chapters are performed using
the atoms and molecules in the optical tweezers.

Chapter \ref{ch:rsc} describes the Raman sideband cooling (RSC) process
we used to cool single Na atom in the optical tweezer.
As the lighter atom with a broader optical linewidth,
the RSC of Na atom faces additional challenges compared to atoms
that were cooled using similar method previously
and the tool we developed to overcome these challenges can be applied to other systems as well.

After preparation of the atomic state,
chapter \ref{ch:interaction-shift} starts our investigation of the interaction between atoms
by measuring the $s$-wave scattering length using interaction shift spectroscopy.
The measurement result is used to refine the prediction for Feshbach resonance
between Na and Cs atoms and also to improve the atomic state preparation.

Chapter \ref{ch:pa} discusses the measurement of the molecular excited states
by photoassociation spectroscopy.
We also study the effect of the tweezer beam on the molecular states and transitions.
The states mapped out in this measurement are used as intermediate states
to study the groud molecular states via two-photon transitions.

Following the previous chapter, chapter \ref{ch:raman-spectroscopy} discusses
the properties of the weakly bound ground molecular states using Raman spectroscopy.
We demostrated the ability to control the rotational states of the molecule
and studies the coupling of the molecule with external field
and the their hyperfine structure in the weakly bound regime.
The measurement identifies candidate target states for our coherent molecule creation.

Chapter \ref{ch:raman-transfer} combines the preparation and characterization
in the previous chapters and describes our all-optical coherent molecule formation process.
A detailed comparison between different approaches and states selection is given
to support our choice of experimental parameters.
We show our result on the coherent transfer and characterizes
the molecule we create as well as the transfer process.
We study the limit on the transfer efficiency
and list open questions in the transfer process
as well as possible future improvements.

% Computer control
% -*- mode: latex-mode; TeX-engine: xetex; LaTeX-command-style: (("" "SOURCE_DATE_EPOCH=0 %(PDF)%(latex) --shell-escape %S%(PDFout)")); TeX-master: "../dissertation.tex"; -*-

\chapter{Computer Control of the Experiment}
\label{ch:computer-control}

\section{Introduction}

The experiment sequence and data taking is managed by computers.
In additional to controlling the timing and the actions during a sequence,
the computer control system is also the main interface
between the people running the experiment (the user),
the data and the hardware performing the manipulation and measurements.
Because of its central role in the experiment,
it has to satisfy many requirements so that
the daily operation of the lab can be performed smoothly and reliably.
\begin{enumerate}
\item Full control and utilization of hardware.\\
  The control system is a layer in between the the user and the hardware
  and will abstract and manage the hardware on behave of the user.
  However, the abstraction must still allow the user to take advantage of
  the full capability of the hardware, e.g. output resolution, timing accuracy etc..
  This is because there is usually little margin between the capability of the hardware
  and the requirement in the experiment as the specification of the hardware
  is often selected based on the requirement to begin with.
\item Usability for all lab members.\\
  The lab is operated by users with specialty in physics rather than computer science.
  Although some basic knowledge of computer programming is required for operating
  the experiment as well as analysing data,
  the computer control system must be fully usable for people without any experience
  in building complex software systems.\\
  The inevitable complexity of the system must be fully hidden from the user
  for normal operations although more direct control may still be allowed in certain cases.
\item Modeling of the sequence and scan.\\
  As an important special case of usability,
  the computer control system must provide a model for each tasks closed to
  the users' mental model.
  More concretely, this means creating abstraction for concepts
  typically used to describe the task and allow operation on these abstractions
  matching the users' expectation.
  We will talk about concrete examples of this requirements
  in section \ref{ch:computer-control:frontend} regarding the sequence frontend
  and section \ref{ch:computer-control:scan} regarding scan automation.
\item Reproducibility.\\
  When exploring something new in the experiment,
  trial and error is the standard method and troubleshooting is a major part of the process.
  The ability to do this effectively requires a high degree of reproducibility of all the result.
  While it is impossible and also not the job of the computer control system to
  eliminate all the fluctuation and noise in the experiment,
  it should not add to the randomness of the system.
  With few exceptions, identical user input should produce identical output from
  the control system.
\item Version control.\\
  As a variation of the reproducibility requirement and also built on top of it,
  we must be able to revert to a previous software configuration at a later time
  in order to reproduce and double-check an earlier result.
  The use of a proper version control system on the settings and code for
  the computer control system can allow this with additional features
  including easy visualization of setting change and parallel development of code
  by multiple users.
\end{enumerate}

The design of the computer control system is mainly guided by these requirements
and we will go into more detail as we describe each part of the system.
The application programming interface (API) provided for sequence and scan programming
are mostly text based due to the flexibility and version control requirement.
Some graphic user interface (GUI) are also included for specific tasks
built on top of the text interface but will not be covered in this chapter.
Section \ref{ch:computer-control:frontend} will cover the frontend of the system
which is used by the user directly to specify an experimental sequence.
Section \ref{ch:computer-control:backend} will discuss the support for various
hardware backends used to run a sequence.
After that, section \ref{ch:computer-control:scan} describes how multiple sequences
can be put together to form a scan and
we will talk about some planned update to the system in section
\ref{ch:computer-control:summary}.

\section{Frontend}
\label{ch:computer-control:frontend}

The frontend is the main user interface of the system to specify a sequence.
Its API is designed around the following concepts.
\begin{enumerate}
\item (Sub-)sequence
\item Time step
\item Channel
\item Pulse
\end{enumerate}

(Abstraction)
(Backward compatibility)
(Flexibility)
(Text based/version control friendly)

\section{Backends}
\label{ch:computer-control:backend}
(communication protocol)
(IR)

\subsection{FPGA Backend}
(clock generation)
(pulse merging)
(compression)

\subsection{NiDAQ Backend}
(Variable clock)

\subsection{USRP Backend}
(SIMD)

\section{Automation of Scan}
\label{ch:computer-control:scan}

(Scan requirement)
(Combination of scans)
(Scope/nested structure)

\section{Summary and Outlook}
\label{ch:computer-control:summary}
(new backend/SPCM)
(native code generation, auto vectorization)
(dynamic logic and dependency tracking/optimization)

% Loading of single atom in optical tweezer
% -*- mode: latex-mode; TeX-engine: xetex; LaTeX-command-style: (("" "SOURCE_DATE_EPOCH=0 %(PDF)%(latex) --shell-escape %S%(PDFout)")); TeX-master: "../dissertation.tex"; -*-

\chapter{Loading of Single Atoms in Optical Tweezer}
\label{ch:loading}

\section{Introduction}
\label{ch:loading:introduction}

\section{Free Spacing Cooling of Atoms}
\label{ch:loading:free-space}

\section{Optical Tweezer}
\label{ch:loading:tweezer}

\section{Loading and Imaging in the Tweezer}
\label{ch:loading:loading}

\section{Summary and Outlook}
\label{ch:loading:summary}

% Raman sideband cooling
% -*- mode: latex-mode; TeX-engine: xetex; LaTeX-command-style: (("" "SOURCE_DATE_EPOCH=0 %(PDF)%(latex) --shell-escape %S%(PDFout)")); TeX-master: "../dissertation.tex"; -*-

\chapter{Raman Sideband Cooling}
\label{ch:rsc}

\section{Introduction}
\label{ch:rsc:introduction}
(In order to achieve full quantum control on molecules, we need to control atoms first.)
(An example of such control)
(Motional degrees of freedom)
(PGC cools to ...)
(RSC to further cool.)
\todo{Define abbreviation RSC}
\todo{Maybe move some intro about large LD parameter here}

\section{Basic Theory}
\label{ch:rsc:basic-theory}

\begin{figure}
  \centering
  \includegraphics[width=8cm]{figures/na_rsc_schematics.pdf}
  \caption[Schematics of Raman sideband cooling for Sodium.]{
    Single Na atom Raman sideband cooling scheme.
    The Raman transitions couples $|2,2;n\rangle$ and $|1,1;n+\Delta n\rangle$
    through the intermediate states $|e_i\rangle$ in the $\mathrm{3^2P_{3/2}}$ electronic states.
    The transitions have a one-photon detuning $\Delta_i\approx75$ GHz.
    Two-photon detuning, $\delta$, is defined relative to the $\Delta n=0$ carrier transition.
    For optical pumping, we use two $\sigma^+$ polarized transitions,
    one to pump the atom state out of $|1,1\rangle$ via $\mathrm{3^2P_{3/2}}$
    and one to pump atoms out of $|2,1\rangle$ via $\mathrm{3^2P_{1/2}}$
    to minimize heating of the $|2,2\rangle$ state.
    \label{fig:rsc:na-schematics}}
\end{figure}

The relevant energy diagram and the laser frequencies for RSC are shown in
Fig.~\ref{fig:rsc:na-schematics}.
We approximate the trapping potential using a harmonic oscillator.
Since this is a separable potential, we can use only the 1D motional state $|n\rangle$
and the result can be easily generalized to the full 3D system.

The cooling sequence consists of two types of pulses.
First, a Raman pulse drives the atom to a different hyperfine state while simultaneously
reduces the motional energy of the atom.
The optical pumping (OP) pulse afterwards then reset the hyperfine state of the atom
and reduce the entropy of the system.
This sequence is then repeated until the system reaches the ground motional state
where there is no more motional energy to be taken out of the system via the Raman pulse.
In this section, we will discuss the theory of each types of pulses individually.
We will cover how the pulses affect cooling performance in section \ref{ch:rsc:challenges}.

\subsection{Raman Transition}
\label{ch:rsc:basic-theory:raman}

As shown in Fig.~\ref{fig:rsc:na-schematics},
the cooling sequence starts with the sodium atom in the
$|s_1\rangle\equiv|2,2\rangle$ hyperfine state,
and a Raman transition is used to drive the atom to the $|s_2\rangle\equiv|1,1\rangle$ state,
where $|F,m_F\rangle$ denotes the $F$ and $m_F$ quantum number for the sodium atom.
The full Rabi frequency for such a transition is given by
\begin{align}
  \Omega_{\mathrm{R}}^0=&\sum_{i}\frac{\Omega_{1i}\Omega_{2i}^*}{2\Delta_i}\label{eq:rsc:basic-theory:raman-rabi}
\end{align}
where the sum is over all the coupled excited states,
$\Omega_{ai}\equiv\langle a|\mathbf{d}\cdot\mathbf{E}_a|e_i\rangle$ is the single photon
Rabi frequency between $|a\rangle$ and $|e_i\rangle$
and $\Delta_i$ is the single photon detuning from excited state $|e_i\rangle$.

In order to account for the motional degrees of freedom, we need to include the spatial
wavefunction of the atom and light into account.
As mentioned above, we approximate the atomic motional wavefunction by the harmonic oscillator
eigenstates $|n\rangle$. Coupling between states different $n$ states from the Raman transition
is allowed due to the recoil from the Raman lasers,
which corresponds to a spacial phase imprinting of $\ue^{\ui\mathbf{\Delta k}\cdot\mathbf{\hat x}}$
where $\mathbf{\Delta k}$ is the wavevector difference between the two Raman beams.
Using the creation ($\hat a^\dagger$) and annihilation ($\hat a$) operators and the relation
$\mathbf{\hat x}=\mathbf{x}_0\paren{\hat a+\hat a^\dagger}$ where $x_0=\sqrt{\hbar/2m\omega}$
is the harmonic oscillator length, the phase factor can be expressed as
$\ue^{\ui\eta^{\mathrm{R}}\paren{\hat a+\hat a^\dagger}}$ where $\eta^{\mathrm{R}}\equiv\mathbf{\Delta k}\cdot\mathbf{x}_0$
is the Lamb-Dicke parameter for the Raman transition.
The matrix element between motional state $|n\rangle$ and $|n'\rangle$ is therefore,
\[ M_{n,n'}=\langle n|\ue^{\ui\eta^{\mathrm{R}}\paren{\hat a+\hat a^\dagger}}|n'\rangle \]
and the final Raman Rabi frequency between motional states $n$ and $n'$ is given by,
\[ \Omega_{\mathrm{R}}^{n,n'}=M_{n,n'}\Omega_{\mathrm{R}}^0 \]
For $n=n'$, this is called a carrier transition and the others are called sideband transitions.
If the final state is higher than the initial one, i.e. $n'>n$, it is a heating sideband.
Likewise, transitions with $n'<n$ are cooling sidebands.

A closed form result for $M_{n,n'}$ is given in \cite{wineland_experimental_1998},
\[ M_{n,n'}=\ue^{\paren{\eta^{\mathrm{R}}}^2/2}\sqrt{\frac{n_<!}{n_>!}}\paren{\eta^{\mathrm{R}}}^{|n-n'|}L_{n_<}^{|n-n'|}\paren{\paren{\eta^{\mathrm{R}}}^2} \]
where $n_<$ and $n_>$ are the lesser and greater, respectively, of $n$ and $n'$,
and $L_n^\alpha$ is the generalized Laguerre polynomial,
\[ L_n^\alpha(x)\equiv\sum_{m=0}^n(-1)^m\begin{pmatrix}n+\alpha\\n-m\end{pmatrix}\frac{x^m}{m!} \]

An important limit is the so-called Lamb-Dicke (LD) regime defined by $\paren{\eta^{\mathrm{R}}}^2(2n+1)\ll 1$.
In this case, we can approximate the phase factor in leading order of $\eta^{\mathrm{R}}$,
\[ \ue^{\ui\eta^{\mathrm{R}}\paren{\hat a+\hat a^\dagger}}\approx1+\ui\eta^{\mathrm{R}}\paren{\hat a+\hat a^\dagger} \]
and the matrix element
\[ M_{n,n'}\approx\delta_{n,n'}+\ui\eta^{\mathrm{R}}\sqrt{n+1}\delta_{n+1,n'}+\ui\eta^{\mathrm{R}}\sqrt{n}\delta_{n,n'+1} \]
the three terms corresponds to the carrier ($n'=n$),
the first order heating sideband ($n'=n+1$)
and the first order cooling sideband ($n'=n-1$) with corresponding strength
$1$, $\eta^{\mathrm{R}}\sqrt{n+1}$ and $\eta^{\mathrm{R}}\sqrt{n}$.
We can clearly see from this approximation that the coupling to other motional state
is stronger for a larger $\eta^{\mathrm{R}}$ and higher motional quantum number $n$.
We will discuss this effect outside the LD regime and its implication
on the cooling performance in more detail in section \ref{ch:rsc:challenges}.

\subsubsection{Scattering from Raman Beams}
\label{ch:rsc:basic-theory:raman-scatter}

In additional to driving the Raman transition, the Raman beams can also cause scattering.
The rate of the scattering is
\footnote{Here we assume that each Raman beam only couples to their respective ground states.
  Including coupling to the other ground state increases the scattering rate but does not change
  the scaling with detuning.},
\begin{align*}
  \Gamma=&\sum_{i}\frac{\Gamma_{ei}\Omega_{1i}^2}{4\Delta_i^2}
\end{align*}
where $\Gamma_{ei}$ is the linewidth of the excited state $|e_i\rangle$.
Together with (\ref{eq:rsc:basic-theory:raman-rabi}), we see that approximately
$\Gamma/\Omega_{\mathrm{R}}\propto1/\Delta$ so a larger detuning should be used
in order to reduce the scattering during RSC.

\subsection{Optical Pumping}
\label{ch:rsc:basic-theory:op}

Driving the system on a cooling sideband with Raman transition can reduce the
motional energy of the atom. However, this is a fully coherent process that does
not reduce the system entropy and is not really ``cooling'' the system
or achieving better control on the quantum state of the system.
Instead, quantum state control is achieved in the RSC via the OP pulse.
The initial hyperfine state $|2,2\rangle$ is a stretched state so it is the
state the system naturally ends up in when $\sigma^+$ light is applied.
However, if this is done using scattering from a $F=2$ to $F'=3$ transition,
the OP beam will allow continuous photon cycle
between the $|2,2\rangle$ and the $|3',3\rangle$ causing unnecessary motional heating during OP.
Therefore, the OP must be done on a $F=2$ to $F'=2$ transition.
Unfortunately, for Na, the corresponding transition
from $\mathrm{3^2S_{1/2}}$ to $\mathrm{3^2P_{3/2}}$
that is used for the MOT is not useable due to the small energy difference of
$60 \mathrm{MHz}$ (or $6$ line widths) between the $F'=2$ and $F'=3$ states
\cite{steck_sodium_nodate}.
Instead, we must use the sodium $\mathrm{D1}$ line,
i.e. $\mathrm{3^2S_{1/2}}$ to $\mathrm{3^2P_{1/2}}$ transition,
which lacks a $F'=3$ excited state.
The $\mathrm{D1}$ light with $\sigma^+$ polarization is only used to pump atoms from
$F=2$ states (in particular $|2,1\rangle$ which is populated during the OP process).
Since the goal of the OP pulse is to clear the atom population in all states but $|2,2\rangle$,
the photon cycling is not a concern for $F=1$ states and the $\mathrm{D2}$ line
is used for OP of $F=1$ states instead.
This also allow us to reuse the MOT light source and simplifies our setup.

\section{Raman Sideband Thermometry}
\label{ch:rsc:thermometry}

From the discussion in section \ref{ch:rsc:basic-theory:raman},
we see that the strength of the sideband transition depends on the initial motional state
as well as the Lamb-Dicke parameter $\eta^{\mathrm{R}}$ of the atom.
This dependency allows us to infer the motional state of the atom
by measuring the sideband height, i.e. the so-called sideband thermometry
\cite{monroe_resolved-sideband_1995,meekhof_generation_1996}.

In particular, for atom with temperature $T$,
the probability for the atom to be in motional state $|n\rangle$ is,
\[ p_n=\frac{\ue^{-n\hbar\omega/k_BT}}{1-\ue^{-\hbar\omega/k_BT}} \]
for a Raman pulse with full Rabi frequency $\Omega_{\mathrm{R}}^0$ and time $t$,
the peak height for the first order heating (+) and cooling (-) sidebands,
\begin{align*}
  h_\pm=&\sum_{n=0}^\infty p_n\sin^2\paren{\frac{\Omega_{\mathrm{R}}^0t}{2}M_{n,n\pm1}}
\end{align*}
note that $p_{n+1}=p_n\ue^{-\hbar\omega/k_BT}$, $M_{n,n'}=M_{n',n}$ and $M_{n,-1}=0$, we have
\begin{align*}
  h_-=&\sum_{n=0}^\infty p_n\sin^2\paren{\frac{\Omega_{\mathrm{R}}^0t}{2}M_{n,n-1}}\\
  =&\ue^{-\hbar\omega/k_BT}\sum_{n=1}^\infty p_{n-1}\sin^2\paren{\frac{\Omega_{\mathrm{R}}^0t}{2}M_{n-1,n}}\\
  =&\ue^{-\hbar\omega/k_BT}h_+
\end{align*}
Therefore, if we measure the ratio of the cooling and heating sideband heights
$\alpha\equiv h_-/h_+$, we can calculate the temperature of the atom with
$\ue^{-\hbar\omega/k_BT}=\alpha$.
The corresponding ground state probability is,
\begin{align*}
  p_0=&\frac{1}{1-\ue^{-\hbar\omega/k_BT}}\\
  =&\frac{1}{1-\alpha}
\end{align*}
We will use this to experimentally characterize the performance of the cooling sequence
in the following sections.

\section{Setup}
\label{ch:rsc:setup}

\begin{figure}
  \centering
  \includegraphics[width=8cm]{figures/na_rsc_geometry.pdf}
  \caption[Beams and field geometry for Sodium Raman sideband cooling]{
    Geometry and polarizations of the Raman and optical pumping beams relative to the
    optical tweezer and bias magnetic field.
    Raman beams R1 and R4 address the radial $x$-mode.
    R1 and R2 address the radial $y$-mode.
    R3 and R4 address the axial $z$-mode, where the beams also couple to radial motion,
    but this coupling can be neglected when the atoms is cooled to the ground state of motion.
    \label{fig:rsc:na-geometry}}
\end{figure}

The geometry of all the beams and field involved is shown in Fig.~\ref{fig:rsc:na-geometry}.
In order to make the cooling more efficient and simplify the sideband thermometry,
we address the motion along the three principle axis of the tweezer using different pairs
of Raman beams.
In order to maximize the beam intensity so that a larger single photon detuning can be used
while maintaining the same Raman Rabi frequency,
we focus the Raman beam onto the single atom with a waist of $\approx100~\mathrm{\mu m}$.
The maximum powers within each Raman beam are between $1$ and $6~\mathrm{mW}$
which give us a maximum Raman Rabi frequency of $50$ to $200~\mathrm{kHz}$.

We apply an external bias magnetic field of $8.8~\mathrm{G}$ parallel to the polarization
of the tweezer beam (and orthogonal to the tweezer beam propagation direction).
This makes the field orthogonal to the effective magnetic field of the tweezer,
which minimizes the vector light shifts~\cite{kaufman_cooling_2012,thompson_coherence_2013}.
Since the optical pumping beam requires $\sigma^+$ polarization,
it is setup to propagate parallel to the applied magnetic field.

\section{Cooling Performance and Challenge with Large Lamb-Dicke Parameter}
\label{ch:rsc:challenges}

\todo{some might belong to introduction of the chapter}
RSC is typically performed in the LD regime where the coupling
to other motional state is small.
Due to the light mass, short wavelength, limitted trap depth and high initial temperature
of the sodium atom however, we have to start our RSC sequence outside the LD regime.
This creates unique challenges to our experiment.
A detailed understanding of the cooling performance is required to understand
and overcome these challenges.

The simplest way to estimate the effectiveness of RSC
is by keeping track of the average energy of the atom during the cooling sequence.
For a typical RSC sequence in the LD regime, all the cooling are done on
the strongest first order cooling sideband.
The energy removed for atom driven in one Raman pulse is therefore, $\Delta E_-=\omega$.
In order to reinitialize the hyperfine state, the sodium atom needs to scatter on average
$2$ photons from the OP pulse which increases the average energy of the driven atom
by $\Delta E_+=4\omega_{\mathrm{recoil}}$
\footnote{The factor of 4 comes from 2 absorbed photons and 2 reemitted photons.}
where $\omega_{\mathrm{recoil}}\equiv \hbar k^2/2m$ is the recoil energy\cite{steck_sodium_nodate}
and $k$ is the OP light wave vector.
The heating to cooling ratio in one RSC pulse cycle is therefore,
\begin{align*}
  \frac{\Delta E_+}{\Delta E_-}=&\frac{2\hbar k^2}{m\omega}=4k^2x_0^2\\
  =&4\paren{\eta^{\mathrm{OP}}}^2
\end{align*}
where $\eta^{\mathrm{OP}}\equiv kx_0$ is the Lamb-Dicke parameter for OP.
Therefore, in order to achieve net cooling, we need $\paren{\eta^{\mathrm{OP}}}^2<0.25$.
In 3D with cooling along multiple axis with different trapping frequency
(and therefore different $\eta^{\mathrm{OP}}$), the $\paren{\eta^{\mathrm{OP}}}^2$ in the requirement
is replaced by a weighted average of different axis depending on the frequency
each axis is cooled in the sequence.

\begin{figure}
  \centering
  \includegraphics[width=\textwidth]{figures/na_rsc_challenges.pdf}
  \caption[Optical pumping motional-state redistribution and Raman coupling]{
    Optical pumping motional-state redistribution and Raman coupling for large LD parameters
    for the axial direction ($z$).
    The range plotted covers $95~\%$ of the initial thermal distribution.
    (A) Motional state distribution after one OP cycle for different initial states motion,
    $n_{\textrm{init}}$.
    Due to photon-recoil and the large LD parameter, $\eta^{\textrm{OP}}_z=0.55$,
    there is a high probability of $n$ changing.
    (B) Matrix elements for Raman transition on the first order cooling sideband
    deviate from $\sqrt{n}$ scaling with multiple minima.
    \label{fig:rsc:na-challenges}}
\end{figure}

In our experiment, the OP Lamb-Dicke parameters are
$\eta^{\mathrm{OP}}_x, \eta^{\mathrm{OP}}_y, \eta^{\mathrm{OP}}_z = 0.25, 0.25, 0.55$.
Based on the metric above, any cooling sequences
that have fewer than $78~\%$ cooling pulse for $z$ (axial) axis, which is generally the case,
should have a net cooling effect.
This, however, does not guarantee cooling into the ground motional state,
nor does it fully characterize the efficiency of the cooling sequence
since the averaging hides a few critical aspect of having a large Lamb-Dicke parameter.

One of the important effects can be seen in Fig.~\ref{fig:rsc:na-challenges}A showing
the motional state distribution after one OP cycle
for different initial motional states $n_{\mathrm{init}}$.
Although the average heating is fixed at $4\omega_{\mathrm{recoil}}$,
independent of $n_{\mathrm{init}}$,
the spread or the uncertainty of $n$ after the OP is significantly
higher for high $n_{\mathrm{init}}$.
This effect significantly increases the difficulty in controlling the state during the
RSC sequence. It can negatively impact the cooling performance and
may lead to increased loss during cooling due to atom escaping to higher motional states.

The other important effect is the dependency of matrix element $M_{n,n+1}$
on the motional level $n$.
While this dependency is not a new effect, since the $\sqrt{n}$ dependency
on the cooling sideband strength exist even in the LD regime
and must be taken into account with pulse time variation
\cite{wineland_experimental_1998,liu_molecular_2019}
to achieve efficient cooling, the high Lamb-Dicke parameter adds even more complications.
As shown in Fig.~\ref{fig:rsc:na-challenges}B, rather than a simple $\sqrt{n}$ dependency,
it is a non-monotonic function and more importantly has multiple minima, so-called ``dead-zone'',
within the range of motional states we are interested in.
The coupling strength for states in the dead-zones can be reduced by more than ten times
which can significant affect the efficiency of the cooling pulse
and even makes it virtually impossible to drive Raman transitions on atoms in these states
in order to cool them further.
A cooling sequence can therefore accumulate pupolations in the dead-zones
rather than the ground state.
Their small coupling strength also reduce their signal level during
Raman sideband spectroscopy making these states nearly invisible to sideband thermometry
which further complicates the optimization of the cooling sequence.

\section{Solution: High Order Sidebands}
\label{ch:rsc:solution-high-orders}

\begin{figure}
  \centering
  \includegraphics[width=8cm]{figures/na_rsc_mele_raman.pdf}
  \caption[Raman coupling including high order sidebands]{
    Matrix elements for Raman transition including high order sidebands.
    During cooling, we utilize the fact that high motional states couple most effectively
    to sidebands with large $|\Delta n|$ in order to overcome the issue with
    variation and dead zone in the coupling strengths.
    \label{fig:rsc:na-mele-raman}}
\end{figure}

The main solution to the issues related to the large Lamb-Dicke parameter
is in fact the large Lamb-Dicke parameter itself.
The increased coupling to other motional state for large Lamb-Dicke parameter
and high motional states applies not only to $|\Delta n|=1$ but to higher $\Delta n$ as well.
Fig.~\ref{fig:rsc:na-mele-raman} shows the coupling to higher order cooling sidebands
which all have comparable strengths as the first order sidebands in different ranges
of motional states.

Because of this, it is now possible, and in some cases preferred, to apply Raman cooling pulse
on the higher order sidebands instead of only the first order one.
These pulses reduce more energy from the system per pulse which directly improves
the cooling to heating ratio and allows better control on the motional state
given the uncertainty after an OP pulse.
More importantly, depending on the motional level, there is always a sideband order
with significant coupling strength that can be used to cool it,
therefore completely removing the coupling dead-zones.
Moreover, by using each sideband orders only near their coupling maxima,
the coupling strength variation is also greatly reduced which removes
the need to vary the pulse times for all but the pulses on the first order sideband.

\section{Solution: Simulation Based Optimization}
\label{ch:rsc:simulation}

\begin{figure}
  \centering
  \includegraphics[width=\textwidth]{figures/na_rsc_sequence.pdf}
  \caption[Simulation optimized Raman sideband cooling sequence for Sodium]{
    Schematic of the cooling pulse sequence. The tweezer is strobed at 3 MHz to
    reduce light shifts during optical pumping~\cite{hutzler_eliminating_2017}.
    Each cooling cycle consists of $8$ sideband pulses.
    The four axial pulses address two sideband orders.
    The two pulses in each radial direction either address $\Delta n=-2$ and $\Delta n=-1$
    or have different durations to drive $\Delta n=-1$,
    at the end of the cooling sequence when most of the population is below $n=3$.
    The Raman cooling and spectroscopy pulses have Blackman envelopes~\cite{kasevich_laser_1992}
    to reduce off-resonant coupling,
    while the measurement Rabi pulses in Fig.~\ref{fig:rsc:na-rabi-flop}
    have square envelopes to simplify analysis.
    \label{fig:rsc:na-sequence}}
\end{figure}

The change in cooling technique by including higher order sidebands, however,
does not remove the effect of coupling variation on the sideband thermometry.
If a non-thermal distribution of motional states is produced by the cooling sequence,
the ratio of the first order sideband height still cannot be trusted to calculate
the temperature or the ground state probability.
Including higher order sidebands in the sideband thermometry could in principle
give us enough information about the state distribution but doing
so for a non-thermal distribution is not easy or reliable.
We therefore use a Monte-Carlo simulation to guide our search
for the optimal sequence \cite{dalibard_wave-function_1992}.
The simulation includes accurate scattering rate and branching ratios
from the tweezer, Raman and OP beams.
For best simulation performance, the atom is assumed to be in a single motional state,
i.e. with a certain $n_x$, $n_y$ and $n_z$, after each Raman or OP step
\footnote{This assumes no coherence between different motional state,
  which is the case as long as each Raman pulses are separated from each other by OP pulses.}.
It is also assumed that each Raman pulse drives only the intended sideband order,
which is a property that needs to be ensured in the experiment
(see section \ref{ch:rsc:implementation:pulse-shaping}).
For each cooling sequence simulated,
the Raman beam power and frequency, and the OP beam power and polarization purity
are varies slightly around the respective expected values
in order to confirm the robustness of the sequence against fluctuation in the experiment.
Fig.~\ref{fig:rsc:na-sequence} demostrate the resulting optimal sequence from the simulation.
In particular, we find that alternating the cooling pulses between two
neighboring orders for the axial direction and $\Delta n=-2$ and $\Delta n=-1$
for the radial directions
eliminates the accumulation of population in motional states with small Raman coupling.
The simulation also confirms that setting the coupling strength of each sideband
to drive a Rabi $\pi$-pulse corresponding to the maximum matrix element motional state
(i.e. the maxima in Fig.~\ref{fig:rsc:na-mele-raman}) yields efficient cooling, initially,
as we expected from section \ref{ch:rsc:solution-high-orders}.
The efficiency of cooling on higher-order sidebands diminishes
as the atom approaches the ground state, so the final cycles utilize only
the $\Delta n=-1$ sideband while alternating between the three axes.

\section{Alignment of Raman and OP Beams}
\label{ch:rsc:alignment}

Due to small waist of the Raman beam, it is important to align the Raman beam to
the single atom with high precision in order to maximize the Raman Rabi frequency
as well as minimizing the intensity fluctuation of the Raman beam experienced by the atom
due to pointing instability.
Such precision cannot easily be achieved using external reference and
must be done by using the single atom itself as the alignment target.

In our experiment, we have developped two different methods to align or verify the alignment
of the Raman beams, both relying on the scattering from the Raman beams.
For initial alignment, or when the beam position is off-center
by more than a beam waist, we couple resonant Sodium D2 light into the Raman beam path
in order to enhance the scattering rate. The course alignment is done based on
maximizing depleting and displacement of the MOT due to radiation pressure.
After that, the fine alignment of the Raman beam is done by reducing the power in the Raman
beam path and maximizing the heating effect on the single atom.
When the Raman beam is focused on the tweezer position,
we can observe a depletion of the single atom live loading signal
while the MOT is not affected as significantly.
This process is then repeated with lower power in the Raman beam path until the desired
position sensitivity is reached.

\todo{appendex about scattering branching ratio calculation}
\todo{make sure spin state detection method (and the ``pushout'' concept) is mentioned}
In order to verify the alignment of the tweezer without any physical change to the beam path,
we use a second method to calibrate the single photon Rabi frequency of the Raman light.
This method requires a working OP to initialize the spin state of the atom
so it is less convenient for aligning from scratch.
To use this method, the atom is first loaded in the tweezer and initialized
in the $|2,2\rangle$ state. We then turn on a single Raman beam at maximum power for a
variable length of time. The off resonance scattering from the Raman beam will cause
the spin state of the atom to change and the population in $F=1$ state is measured
by removing the $F=2$ population using a pushout pulse.
The rate of the spin change is fitted to a theory model to derive the Rabi frequency of
the Raman light. This measurement shows that the Rabi frequencies for the Raman beams are
... \todo{number, maybe table}. We can also calculate the scattering rate from the Raman
beams to be ... \todo{number, maybe table} which corresponds to a total of $...$ \todo{number}
scattering event on average during the whole cooling sequence and
should not be a limiting factor for the cooling performance.

The OP beam has a much larger waist ($\approx1~\mathrm{mm}$)
and therefore require less alignment in the beam position.
However, in order to take advantage of the dark state optical pumping and
minimize unnecessary scattering for atoms in the $|2,2\rangle$ state,
the OP beam need to have a pure $\sigma^+$ polarization.
This requires the OP beam to propagate parallel to the magnetic field
in additional to having the currect circular polarization.
The alignment is done by minimizing ``depumping'' of the atom spin state caused by the OP beam,
similar to the technique we used to calibrate the Raman beam Rabi frequency.
After the atom is initialized in the $|2,2\rangle$ state, we turn on the D1 OP light
for a certain time which should not address the atom when perfectly aligned.
The misalignment of the beam, however, will cause the atom to scatter from the OP beam
and change to $F=1$ state, i.e. ``depumped'', with certain probability.
We then change the alignment of the beam an minimize the depumping rate.
Due to a similar requirement for Cesium OP, the probagation direction of
the OP beam for Sodium and Cesium are aligned to each other by overlapping them
using mechanical target to better than $0.08^\circ$ first
before the magnetic field direction is aligned to the OP beams by minimizing depumping.
For polarization alignment, we first clean up the linear polarization of the light
using a Thorlabs nanoparticle linear film polarizer\todo{part number} with better than $100,000:1$
extinction ratio. After that we use both a half waveplate and a quater waveplate
to generate a circularly polarized light.
We observed that both the polarization cleanup and the half waveplate is necessary
to obtain the best polarization alignment in order to compensate for the polarization
fluctuation caused by the fiber as well as the birefringence of the optics and windows
within the OP beam path. After alignment, the OP intensity is calibrated
by measuring the OP rate for atom prepared in $|2,1\rangle$ state
using Raman transitions\footnote{One from $|2,2\rangle$ to $|1,1\rangle$
  and a second one from $|1,1\rangle$ to $|2,1\rangle$.}.
From this measurement we determined that the purity of the OP polarization
to be ... \todo{number}.
\todo{decide if thin film polarizer should be mentioned earlier}

Other than the alignment procedure above, we have also observed that reflection
of the OP beam can contribute significantly to the polarization impurity and
must be avoided. In particular, since the Raman beam R1 counter propagate with the OP beam,
it is possible for the OP beam to be coupled into the Raman fiber and then retro-reflected
to be focused onto the atom through the Raman beam path at a wrong polarization.
Since the Raman beam size is much smaller, we have observed as much as $3~\%$
polarization impurity caused by this mechanism despite only a small amount of
power being reflected. This issue, along with other reflections,
are reduced by avoiding optics with normal incident on the exit path of the OP beam
as well as changing the propagation direction of the R1 Raman beam
to have a small angle with the OP beam which reduces the OP power coupled into
the Raman beam fiber.
\todo{Make sure that the importance of tweezer switching on OP is mentioned before this}

\section{Implementing Optimized RSC Sequence}
\label{ch:rsc:implementation}

In order to achieve the optimal cooling performance, a few more considerations are important
for implementing the sequence from section \ref{ch:rsc:simulation}.

\subsection{Pulse shaping}
\label{ch:rsc:implementation:pulse-shaping}

In order to achieve optimal performance from the cooling sequence,
it is important to accurately drive the intended sideband order.
In fact, in the absense of undesired scattering,
the frequency resolution of the Raman transition limits the lost achievable temperature.
This is particularly important when driving the first order cooling sidebands
since any coupling to the carrier may change the spin state of the atom without
removing any motional energy therefore causing a net heating effect after the OP pulse.

The obvious way to achieve this is to narrow the linewidth of the Raman transition,
e.g. by using a lower power or Raman Rabi frequency.
However, reducing the lineiwth of the transition also increase the susceptibility
to resonance fluctuation.
Therefore the desired solution is to reduce the off-resonance coupling of the Raman beam
for large detuning while increasing or maintaining the coupling for small detuning.
We achieve this by using a Blackman pulse shape for the Raman transition
\cite{kasevich_laser_1992}.
\footnote{While more complex pulses can be constructed to achieve a even sharp detuning cutoff,
  \todo{\cite{}} such pulses generally significantly increases the pulse time
  and can cause more heating during cooling due to scattering and other heating mechanisms.
  The Blackman pulse we use offers a balance between the pulse time
  and off-resonance coupling reduction.}

\subsection{Calibration}
\label{ch:rsc:implementation:calibration}

The Raman sideband frequencies are calibrated by measuring the Raman spectrum before cooling
(an example of which is shown in the initial spectrum in Fig.~\ref{fig:rsc:na-spectrum}).
However, since high sideband orders are mainly used to cool atoms in high motional state,
the resonance frequency for these sideband are not equal spacing anymore due to
the anharmonicity of the trap.
In order to estimate the effect on the sideband frequencies,
we can define anharmonicity as $A_{i,n}=(E_{i,n+1}-E_{i,n})/h - \omega_i/(2\pi)$
for each trap axis $i$, and calculated from the quartic term
of the optical tweezers via perturbation theory.
In the paraxial approximation, we find $A_{i,n}=\frac{-3n\hbar}{4\pi m d_i^2}$,
where $d_i$ equals the beam radius for the radial directions and
$d_z\approx\pi w_{0,x}w_{0,y}/\lambda_{\textrm{trap}}$.
Numerically, $\{A_{x,n},A_{y,n},A_{z,n}\}=\{-1.4, -1.4, -0.16\}n~\mathrm{kHz}$.
For the states addressed by high order sidebands,
this broadens and shifts high-order sidebands
due to the $n$-dependence of the transitions.

To mitigate this, we calibrate the frequency of each sideband order individually
at the initial temperature. However, since the first order sideband is mainly used to cool atoms
that are closed to the ground state,
their resonance frequency  is recalibrated using partially cooled atoms after the initial calibration.
The use of high Rabi frequency and Blackman pulse shape also reduces the effect
of anharmonicity by broadening the spectrum as much as possible.

Although we can calibrate the single photon Rabi frequency of the Raman beams from the scattering rate
(section \ref{ch:rsc:alignment}), we also calibrate the Raman Rabi frequency on the carrier and
different sideband orders.
This offers a more direct and sensitive measurement for the cooling sequence parameter.
Unlike resonance frequency, the anharmonicity only has a second order effect on the Rabi frequency
and is therefore ignored. The calibration measures
only the carrier and first order heating sideband which has the highest signal after cooling
(an example of which is shown in the cooled Rabi flopping signal in Fig.~\ref{fig:rsc:na-rabi-flop}).
The Raman Rabi frequencies measured on these two transitions are used to calculate
the full Rabi frequency $\Omega_{\mathrm{R}}^0$ and the Lamb-Dicke parameter,
which are in-turn used to calculate the Rabi frequency on other sideband orders.

Since the final calibration of both the Raman Rabi frequency and resonance requires a working
cooling sequence, when optimizing the sequency from scratch,
the calibration process is applied iteratively as the cooling performance is improved.

\section{Cooling Performance}
\label{ch:rsc:performance}

\begin{figure}
  \centering
  \includegraphics[width=\textwidth]{figures/na_rsc_spectrum.pdf}
  \caption[Raman sideband spectra before and after cooling]{
    Raman sideband spectra for (A) $x$, (B) $y$, (C) $z$ axis before (red circle)
    and after (blue square) applying Raman sideband cooling sequence.
    The height of the cooling sidebands (positive detuning)
    are strongly suppressed after cooling which suggests most of the atoms are cooled
    to the motional ground state in the trap.
    \label{fig:rsc:na-spectrum}}
\end{figure}

Our final cooling results are shown in Fig.~\ref{fig:rsc:na-spectrum} and
\ref{fig:rsc:na-rabi-flop}.
In total, $540$ cooling pulses (total duration $53~\mathrm{ms}$) are applied
along three axes with cooling beginning on the radial second order and axial fifth order.
The full sequence including calibrated parameters can be found in appendix \ref{appendex:rsc}.

\todo{
  To characterize the single atom thermal state before and after cooling,
  we perform Raman sideband thermometry~\cite{Monroe1995, Meekhof1996}.
  For the more tightly confined radial directions,
  we observe clear $\Delta n=1$, $\Delta n=-1$, and $\Delta n=-2$ sidebands before RSC,
  as is shown in Fig.~\ref{f-radial}A.
  After RSC, the $\Delta n=-1$ and $\Delta n=-2$ sidebands on both radial axes are strongly reduced.
  The formula that equates the ratio of $\Delta n=-1$ and $\Delta n=1$ sideband heights to
  $\bar{n}/(\bar{n}+1)$ ($\bar{n}$ is the mean motional quantum number of the assumed thermal state)
  was derived in the LD regime~\cite{Monroe1995}.
  However, it is also valid outside the LD regime.
  The resulting $\bar{n}_x=0.019(4)$ and $\bar{n}_y=0.024(3)$ correspond to
  ground-state fractions of 98.1(5)\% and 97.6(3)\%,
  in agreement with fitted values of 98(1)\% and 95(3)\%
  from the Rabi flopping curves~\cite{Meekhof1996} in Fig.~\ref{f-radial}B and \ref{f-radial}C.
  The initial temperature of $80\,\mu$K before RSC is obtained from similar fits.

  For the weak axial direction, cooling is challenging because the atom starts outside the LD regime.
  We observe up to 5th-order Raman cooling sidebands initially,
  which indicates population in highly-excited motional states.
  Nevertheless, our cooling sequence works efficiently as all the $\Delta n<0$ sidebands are reduced
  after RSC (Fig.~\ref{f-axial}A).
  Using the ratio of first-order sideband heights, we obtain $\bar{n}_z=0.024(5)$,
  which corresponds to a ground state population of
  97.6(5)\%, in agreement with a ground state population of 95(4)\% extracted from Rabi flopping
  when $\Delta n=0$ (Fig.~\ref{f-axial}B).
  For the $\Delta n=1$ sideband (Fig.~\ref{f-axial}C),
  we observe additional decoherence that is more pronounced due to the slower Rabi frequency.
  The decoherence rate is consistent with magnetic field fluctuations of $1.5$ mG
  measured independently in the lab, which would produce a Zeeman shift of $\sim 3$ kHz.

  Combining the axial and radial cooling results,
  a single Na atom is in the 3D ground state with a probability of 93.5(7)\% after RSC.
  The cooling efficiency is limited by spontaneous scattering rate
  (0.1-0.2 kHz) from the Raman beams,
  as well as spectral broadening from magnetic field fluctuations.

  We measure a heating rate that corresponds to decreasing 3D ground state population
  at a rate of $\sim0.9$\%/ms.
  The rate is consistent with off-resonant scattering of the trapping light~\cite{Grimm2000},
  and is predominantly in the axial direction where the trapping beam propagates.

  Monte-Carlo simulations show that the ground state probability after RSC
  could be enhanced by increasing the detuning of the Raman beams and implementing
  better control of the magnetic field. Another improvement could come from
  grey molasses cooling, to achieve a lower starting temperature before RSC~\cite{Colzi2016}.

  We have shown that despite the difficulty in achieving a low optical cooling temperature
  of low mass sodium atoms, three dimensional cooling
  with significant ground state population can be achieved by using high-order Raman sidebands
  in an optimized cooling sequence.
  These techniques are well-suited for a large variety of systems
  and open up a route to ground state cooling for other species,
  including molecules and exotic atoms.
}

\begin{FPfigure}
  \includegraphics[width=\textwidth]{figures/na_rsc_rabi_flop.pdf}
  \caption[Rabi flopping on carriers and sidebands]{
    Rabi flopping on radial axis $x$ (A) carrier and (B) $\Delta n_x=1$ sideband,
    radial axis $y$ (C) carrier and (D) $\Delta n_x=1$ sideband,
    axial axis $z$ (E) carrier and (F) $\Delta n_x=1$ sideband,
    before (red circle) and after (blue square) Raman sideband cooling.

    Solid lines (both red and blue) in all plots are fits to a Rabi-flopping
    that includes a thermal distribution of motional states~\cite{meekhof_generation_1996}
    as well as off-resonant scattering from the Raman beams.

    The blue lines correspond to a ground state probability of (A-D) $98.1~\%$ along radial axis
    and (E-F) $95~\%$ along the axial axis after cooling.
    The red lines correspond to a thermal distribution of $80~\mathrm{\mu K}$ before RSC.
    The horizontal dashed lines in all the plots correspond to the $4~\%$ probability
    of imaging loss.

    The green dashed line in (F) includes the additional decoherence due to
    a fluctuation of the hyperfine splitting of magnitude $3~\mathrm{kHz}$.
    We see that the decoherence effect is strongest for the post-cooling data on
    the axial $\Delta n_z=1$ sideband where the Rabi frequency is the lowest.
    \label{fig:rsc:na-rabi-flop}}
\end{FPfigure}

\section{Summary and Outlook}
\label{ch:rsc:summary}

\todo{
  mention merging
  other systems
}

% Interaction of single atoms
% -*- mode: latex-mode; TeX-engine: xetex; LaTeX-command-style: (("" "SOURCE_DATE_EPOCH=0 %(PDF)%(latex) --shell-escape %S%(PDFout)")); TeX-master: "../dissertation.tex"; -*-

\chapter{Interaction of single atoms}

\section{Scattering length}
(Importance/relation with binding energy etc.)

\section{Energy levels of two interacting atoms in an anisotropic trap}

\section{Interaction shift spectroscopy}
(motional sideband, scattering length result)

% Photoassociation of single atoms
% -*- mode: latex-mode; TeX-engine: xetex; LaTeX-command-style: (("" "SOURCE_DATE_EPOCH=0 %(PDF)%(latex) --shell-escape %S%(PDFout)")); TeX-master: "../dissertation.tex"; -*-

\chapter{Photoassociation of Single Atoms}
\label{ch:pa}

\section{Introduction}

\todo{
  Need to measure the excited states in order to drive a two photon transition
  through the excited state.
}

\section{Energy Levels}

First we will discuss the energy levels in a diatomic molecule as well as the labeling system.
We will focus mainly on the electronic excited states measured in this chapter
but most of the discussion here applies to ground electronic states as well
and will be useful for chapter \label{ch:raman-spectroscopy} and \label{ch:raman-transfer}.

\todo{FCF}

\todo{
\item Also mostly about the interaction at short range/state with large binding energy.

  {
  \item Has many degrees of freedoms (in additional to twice the number of electron orbit,
    electron spin and neuclear spin, also neuclear motion)
  \item Important to consider the symmetry, especially the angular momentum
  \item Strongest from S1-S2
  \item LS coupling, hunds case a
  }



  {
  \item Electronic (including spins via exchange interaction)
  \item Vibrational
  \item Rotations (coupled to the spin)
  \item HF, will be ignored for the most part for the excited state
    We will breifly discuss the ground state HF structure in next chapter
  }
  {
  \item BO approximation, motivated by the mass ratio between the nuclear and electron mass.
    Allows us to treat the electronic energy as a function of nuclear motion (PEC).
    Formally the approximation corresponds to the energy gap between electronic state
    to be much larger than neuclear motion, thus the approximation breaks down when
    electronic state energy crosses.
    For most of our discussion, we operate far away from the crossing of the electronic
    potentials and therefore the approximation is justified.
  }
}

\section{Effect of the Trap}

(light shift, broadening)

\section{Photoassociation Spectroscopy}
(v=0, 12, 14, etc)

% Two-photon spectroscopy of NaCs ground state
% -*- mode: latex-mode; TeX-engine: xetex; LaTeX-command-style: (("" "SOURCE_DATE_EPOCH=0 %(PDF)%(latex) --shell-escape %S%(PDFout)")); TeX-master: "../dissertation.tex"; -*-

\chapter{Two-photon Spectroscopy of NaCs Ground State}
\label{ch:raman-spectroscopy}

\section{Introduction}

\todo{make sure the goal of reaching the absolute ground state is stated previously.}

The excited molecular states measured and characterized in chapter \ref{ch:pa}
provide us a pathway to couple to the ground electronic molecular states
using two-photon transitions.
While it is in principle possible to drive from the atomic state
to any desired molecular ground state for various applications,
doing so directly has many technical challenges.
We will cover these challenges as well as the considerations in selecting
the molecule formation pathway in chapter \ref{ch:raman-transfer},
however, the difficulty, and the main difference from a pure atomic Raman transition,
lies in the wavefunction size mismatch.
As we have seen already in section \ref{pa:beampath},
the size mismatch between the excited molecular states and the ground atomic state
causes a very small FCF and requires a high PA intensity to improve the signal strength.
This small FCF also reduces the Rabi frequency for the two-photon transition to the ground state.
As a result, driving a two-photon transition to an arbitrary molecular ground state
may require maintaining coherence between two different lasers over
a relatively long time (milliseconds) which is very difficult to achieve.

Our solution to this challenge is to do the transfer via a two step process.
\begin{enumerate}
\item We drive a two-photon transition from the atomic state to
  a weakly bound ground state.
  The reduced energy difference allows the laser coherence to be maintained
  over a longer time easily.
  In the case of NaCs molecule, this also increases the FCF
  between the ground and excited molecular states which allows shorter pulse time and
  further reduces the coherence requirement.
\item The transfer to arbitrary molecular ground state will be done from
  the weakly bound state created in the first step.
  The strength of this transition can be much higher
  and only requires a relatively shorter laser coherence time.
  This step has already been demonstrated in other experiments\todo{\cite{}}
  so in this thesis we will focus only on the first step transfer.
\end{enumerate}

In this chapter, we will discuss the use of Raman spectroscopy
to measure the properties of the weakly bound molecular ground states.
In section \ref{ch:raman-spectroscopy:states}
we will describe the states involved and the setup for the Raman spectroscopy
as well as the measured binding energy for the $N=0$ states.
In section \ref{ch:raman-spectroscopy:n2}
we study the coupling between angular momentums for near threshold molecular states
by characterizing the $N=2$ states.

\section{Weakly bound NaCs Ground States}
\label{ch:raman-spectroscopy:states}

As mentioned in section \ref{pa:structure:near-threshold},
the angular momentum coupling for weakly bound molecular state is similar to that of the atoms.
Therefore, instead of using the term symbol for the \textit{Hund's case (a)}
to identify the molecular potential and bound states,
we use the hyperfine state ($F_{Na}, F_{Cs}$) for the atoms instead.

\todo{figure?}
In order to measure the binding energy of a molecular state,
we first prepare the atom in the corresponding hyperfine state
and drive a Raman transition to the molecular state.
We use a Raman transition that is detuned from the $c^3\Sigma\ v'=0$ state measured
in section \label{pa:pa}.
\todo{Maybe move stable HF combination to interaction shift?}
When the Na and Cs are in the same tweezer,
they can undergo fast spin-exchange collision that changes the hyperfine state of the atom.
This process can cause the hyperfine energy ($>1\mathrm{GHz}$) of the atoms
to be transferred to the motional energy
and eject the atoms from the tweezer ($<100\mathrm{MHz}$ deep).
As a result, the measurement can only be done when the spin-exchange collision is suppressed,
which includes the following spin combinations,
\begin{enumerate}
\item $F_{Na}=1$ and $F_{Cs}=3$\\
  This is the spin state with the lowest energy and therefore the spin-exchange interaction
  is energetically forbidden.
  In the experiment, we use the state $|Na(1, 1),Cs(3, 3)\rangle$
  which can be prepared from the $|Na(2, 2),Cs(4, 4)\rangle$ state from OP
  easily by driving a Raman transition for both Na and Cs atoms.
  This state also remains the lowest energy atomic state in the present of a weak magnetic field.
\item $|Na(2, 2),Cs(4, 4)\rangle$ and $|Na(2, 2),Cs(3, 3)\rangle$
  \footnote{States with opposite $m_F$, i.e.
    $|Na(2, -2),Cs(4, -4)\rangle$ and $|Na(2, -2),Cs(3, -3)\rangle$ are also stable
    but are omitted here since these cannot be easily prepared in our experiment.}\\
  These spin states are stable because the spin-exchange collision conserves total $m_F$
  of the two atoms and
  the two states are the lowest energy states that has the same total $m_F$.
  Inelastic collision that changes the total $m_F$ can also happen
  but has a lower collision rate since it requires transferring angular momentum
  between the spin and motion of the atom.
\end{enumerate}

\subsection{Driving Raman Transition using the Optical Tweezer}

\begin{figure}
  \centering
  \includegraphics[width=\textwidth]{figures/raman_spectroscopy_raman_beampath.pdf}
  \caption[Beampath to allow driving Raman transition with tweezer]{
    Beampath for generating the frequency for Raman transition in the tweezer.
    (Beampath for fiber coupling and overall power control is not shown.)
    The red beam path is the $0$-th order of the double pass (DP) AOM
    which is used for the tweezer.
    When the DP AOM is turned on, some power is redirected to the first order
    (blue beam path) which generates the required frequency different to drive
    the Raman transition. The two frequencies are recombined on the DP AOM.
    The $0$-th order light is shifted by another single pass (SP) AOM
    running on a different frequency before recombining.
    Without this AOM, the leak light from the DP AOM will be at the same frequency
    as the $0$-th order light which can cause a significant power fluctuation
    due to interference. The SP AOM ensures that none of the leaking light frequency
    coincide with either intended frequencies therefore avoiding this issue.
    Different selection of the SP and DP AOM as well as their orders can be used
    to cover a wide range of two photon detuning for Raman transition.
    The experiment typically start with the SP AOM on and the DP AOM off.
    When driving the Raman transition, the powers on both AOMs are ramped simultaneously
    to achieve the desired power at both frequencies.
    \label{fig:raman-spectroscopy:raman-beampath}}
\end{figure}

In order to increase the intensity of the Raman beams to overcome the small FCF,
we use the tweezer beam to drive the Raman transition.
Not only does this maximizes the intensity due to the small focal size of the tweezer,
since the atoms are trapped at the maximum of the tweezer beam,
this also ensures that the Raman beam is aligned automatically to the atoms
and suppresses sensitivity to mechanical fluctuation that is usually
caused by a small beam size.
Moreover, this also minimizes the number of beams the atoms experience
during the Raman transition which, in turns, minimizes the scattering.
As a result, the coherence of the transition is also improved
which is important for achieving coherent creation of molecule
(chapter \ref{ch:raman-transfer}).

The beampath to generate the required frequencies in the tweezer is shown in
Fig.~\ref{fig:raman-spectroscopy:raman-beampath}.
Care must be taken to avoid interference of beam causing power fluctuation.

\todo{
  sequence (instead of turning on the Raman beam, turn on the second frequency as seen in)
  calibration?
}

\subsection{Raman Resonance on $v''=-1,\ N=0$ Ground State}
\todo{
  N=0 bound states for different HF states
  B field dependency?
}

\section{Angular momentum Coupling in $N=2$ Ground State}
\label{ch:raman-spectroscopy:n2}

\todo{
  N=2 bound states/field dependency
}
\todo{
  (NaCs 2017-2018 > NaCs 2018 > Raman transfer 10/5 - 10/8 > v=-1 N=2 properties 11/03 - 11/07)
}

% Coherent optical creation of NaCs molecule
% -*- mode: latex-mode; TeX-engine: xetex; LaTeX-command-style: (("" "SOURCE_DATE_EPOCH=0 %(PDF)%(latex) --shell-escape %S%(PDFout)")); TeX-master: "../dissertation.tex"; -*-

\chapter{Coherent Optical Creation of NaCs Molecule}
\label{ch:raman-transfer}

\section{Introduction}

\todo{emphasize general approach}

\section{Raman Transition Beyond Three-Level Model}

In an ideal three-level system, the scattering probability during a $\pi$ pulse
Raman transition can be made arbitrarily small by using a large single photon detuning.
However, in a real system, there are often other effects that increases the scattering
and may also put a lower limit on the scattering probability during the transfer.
Fig.~\ref{fig:raman-transfer-generic-raman-model} shows a generic model
for a real Raman transition demostrating some of these effects.
Additionally, other practical limitation in the system like stability of the laser power
and frequency also needs to be taken into account.

In the experiment, we find the parameter range that gives the best transfer efficiency
using numerical simulation (see section \ref{ch:raman-transfer:state-selction}).
Nevertheless, in order to develop a general approach that can be applied to other systems,
it is also important to understand the various physical mechanism that leads
to the optimal parameters.
Therefore, in this section, we will discuss some of the most important effects
on the transfer efficiency at qualitative and semiquantitative level.
Due to experimental constraint, we will assume that the single photon detuning is
much smaller than the frequency of each individual beams, i.e. $\Delta\ll\nu_1,\ \nu_2$.

\begin{figure}
  \centering
  \includegraphics[width=0.6\textwidth]{figures/raman_transfer_generic_raman_model.pdf}
  \caption[Generic model for a real Raman transition]{
    Generic model for a real Raman transition.
    The initial state $|i\rangle$ and the final state $|f\rangle$
    has a energy difference $\delta$
    and are coupled by two Raman beams with frequencies and
    single photon Rabi frequencies of $\nu_1,\ \Omega_1$ and $\nu_2,\ \Omega_2$ respectively.
    The corresponding matrix elements (arbitrary unit) are $M_1$ and $M_2$.
    The Raman beams are detuned by $\Delta$ from the primary excited state $|e\rangle$,
    which has a decay rate of $\Gamma_e$.
    We also consider additional states near the initial ($|i'\rangle$),
    final ($|f'\rangle$) and intermediate excited $|e'\rangle$ states which are
    separated from the corresponding Raman transition states by $\omega'_i$,
    $\omega'_f$ and $\omega'_e$ respectively.
    Only one additional state of each kinds are included to simplify the discussion
    without loss of generality.
    \label{fig:raman-transfer-generic-raman-model}}
\end{figure}

\subsection{Additional Initial and Final States}

First, we will discuss the effect of $|i'\rangle$ and $|f'\rangle$ states
near the initial and final states.
These states can be coupled to the excited state $|e\rangle$ by the Raman beams,
which can in tern be coupled to the initial and final states
by an off-resonance Raman transition.
The leakage is suppressed by the detuning from the Raman resonance,
i.e. $\omega'_i$ and $\omega'_f$.
This puts a limit on the Raman Rabi frequency $\Omega_R$ to be smaller
than the smallest energy gap, which in turns puts a limit on the minimum Raman transfer time.
In our experiment, the minimum energy gap comes from axial motional excitation of
the atomic initial states which is between $2\pi\times10 - 30$~kHz
depending on the trap depth used.
The typical Raman $\pi$ time we can realize is $0.5 - 5$~ms so this effect
is not a major limiting factor for our transfer efficiency.

\subsection{Additional Excited states}
\label{ch:raman-transfer:extra-ext}

\begin{figure}
  \centering
  \includegraphics[width=\textwidth]{figures/raman_transfer_extra_ext_states.pdf}
  \caption[Raman transition with additional excited states]{
    Effect of additional excited states $|e'\rangle$ on the Raman transition efficiency.
    (A) Depending on the sign of the coupling, there could be constructive (blue)
    or destructive (orange) interference on the Raman Rabi frequency $\Omega_{Raman}$.
    (B) Increased scattering rate $\Gamma_{scatter}$ caused by $|e'\rangle$ with a minimal
    between the two states.
    (C) Optimal detunine exists between the two states with maximum transfer efficiency
    corresponds to a fraction of the state spacing.
    \label{fig:raman-transfer-extra-ext-states}}
\end{figure}

Next, we will consider the effect of the $|e'\rangle$ state near the excited intermediate state.
These states can be coupled to the ground states, both $|i\rangle$ and $|f\rangle$,
by the Raman beams and can cause a change in both the Raman Rabi frequency
and the scattering rate.
The total Raman Rabi frequency (Fig.~\ref{fig:raman-transfer-extra-ext-states}A) is,
\[
  \Omega_{Raman}=\frac{\Omega_1\Omega_2}{2\Delta}+\frac{\Omega'_1\Omega'_2}{2(\Delta-\omega'_e)}
\]
where $\Omega'_1$ and $\Omega'_2$ are the single photon Rabi frequencies coupling $|e'\rangle$
to $|i\rangle$ and $|f\rangle$ respectively.
Depending on whether $\Omega'_1\Omega'_2$ has the same (orange line)
or different (blue line) sign as $\Omega_1\Omega_2$, the total Raman Rabi frequency
may be cancelled or enhanced between the two excited states.
On the other hand, the total scattering rate (Fig.~\ref{fig:raman-transfer-extra-ext-states}B)
is almost always increased due to the additional state, creating a local minimum
between the excited states.
Combining the two effects, the ratio between the Raman Rabi frequency and
the scattering rate, which determines the transfer efficiency, always have local maximum
between the excited states (Fig.~\ref{fig:raman-transfer-extra-ext-states}C).

Despite the difference in the position and value of the maximum for different
$|e'\rangle$ parameters, we can summarize the effect on the transfer efficiency
as a limit on the maximum detuning $\Delta_{max}$ to a fraction of the spacing
between the excited states ($\omega'_e$).
As an example, the blue and orange maxima in Fig.~\ref{fig:raman-transfer-extra-ext-states}C
corresponds to a limit on single photon detuning of $0.5\omega'_e$ and $0.15\omega'_e$.
As one would expected, a larger excited state spacing usually result in
a larger detuning limit and a better transfer efficiency.

Summarizing the effect of additional excited state as a single number $\Delta_{max}$ allows
us to keep using the equation for Raman transition with minor corrections
and makes it easier to compare different state selection and transition schemes.
It is also worth noting that although only one additional excited state $|e'\rangle$
is considered here, this result can be generalized when more excited states are taken into account
as well. These states introduces additional smooth variation in both the Raman Rabi frequency
and scattering rate and the effects on the final transition efficiency can be similarly
treated as a change in the maximum detuning.

\subsection{Cross Coupling Between Light Addressing Initial and Final States}

Due to the small energy separation between the initial and final state $\delta$,
the cross coupling of the laser addressing the initial/final state on the final/initial state
is another important effect in our experiment.
Without the cross coupling, the total off resonance scattering rate for
the initial and the final states is
\[
  \Gamma_{scatter0}=\frac{\Gamma_e\left(\Omega_1^2+\Omega_2^2\right)}{2\Delta^2}
\]
For a given Raman Rabi frequency $\Omega_{Raman}\propto\Omega_1\Omega_2$, this is
minimized when $\Omega_1=\Omega_2$.\todo{reference}

When cross coupling is taken into account, however, the total scattering rate becomes,
\footnote{Here we assume that the matrix elements are the same for the two beams.
  This is the case when the two beams have the same polarization as in our experiment.
  This effect can be minimized or eliminated by selecting different polarizations for the
  two laser frequencies that does not couple to the other initial/final state.
  This would also require choosing an excited state with the same or lower angular momentum
  as the ground states in order to avoid cross coupling to different excited states.}
\begin{align}
  \Gamma_{scatter}=&\frac{\Gamma_e\Omega_1^2}{2M_1^2}\left(\frac{M_1^2}{\Delta^2}+\frac{M_2^2}{(\Delta+\delta)^2}\right)+\frac{\Gamma_e\Omega_2^2}{2M_2^2}\left(\frac{M_2^2}{\Delta^2}+\frac{M_1^2}{(\Delta-\delta)^2}\right)\\
  \propto&\frac{\Gamma_eP_1}{2}\left(\frac{M_1^2}{\Delta^2}+\frac{M_2^2}{(\Delta+\delta)^2}\right)+\frac{\Gamma_eP_2}{2}\left(\frac{M_2^2}{\Delta^2}+\frac{M_1^2}{(\Delta-\delta)^2}\right)
\end{align}
Where $P_1$ and $P_2$ are the powers of the laser beams 1 and 2.
When $\delta\ll\Delta$ such as our experiment,
\begin{align*}
  \Gamma_{scatter}\approx&\frac{\Gamma_e\left(M_1^2+M_2^2\right)}{2\Delta^2}\left(\frac{\Omega_1^2}{M_1^2}+\frac{\Omega_2^2}{M_2^2}\right)\\
  \propto&\frac{\Gamma_e\left(M_1^2+M_2^2\right)}{2\Delta^2}\left(P_1+P_2\right)
\end{align*}
For a given Raman Rabi frequency $\Omega_{Raman}\propto\Omega_1\Omega_2\propto\sqrt{P_1P_2}$,
this is minimized when $P_1=P_2$.
Hence, due to the strong cross coupling, we need to use the same power in both Raman beams
rather than adjusting the powers to match their single photon Rabi frequencies.

Moreover, at the minimum scattering rate, we have $\Omega_2=\Omega_1M_2/M_1$
and the ratio between Raman Rabi frequency and scattering rate is,
\begin{align*}
  \frac{\Omega_{Raman}}{\Gamma_{scatter}}=&\frac{\Omega_1\Omega_2}{2\Delta}\frac{2\Delta^2}{\Gamma_e\left(M_1^2+M_2^2\right)}\left/\left(\frac{\Omega_1^2}{M_1^2}+\frac{\Omega_2^2}{M_2^2}\right)\right.\\
  =&\frac{\Delta\Omega_1\Omega_2}{\Gamma_e\left(M_1^2+M_2^2\right)}\left/\left(\frac{\Omega_1^2}{M_1^2}+\frac{\Omega_2^2}{M_2^2}\right)\right.\\
  =&\frac{\Delta\Omega_1^2M_2}{2\Gamma_eM_1\left(M_1^2+M_2^2\right)}\frac{M_1^2}{\Omega_1^2}\\
  =&\frac{\Delta}{2\Gamma_e}\frac{M_1M_2}{M_1^2+M_2^2}
\end{align*}
Therefore, for a given excited state linewidth $\Gamma_e$
and maximum detuning (section \ref{ch:raman-transfer:extra-ext}) the transfer efficiency
maximizes for the smallest $M_1M_2/(M_1^2+M_2^2)$ which happens when the ratio
$M_1/M_2$ is the closest to $1$.

The light shift of the Raman resonance is similarly affected by the cross coupling.
The differential light shift between the initial and the final state determines
the resonance fluctuation as a function of light intensity fluctuation.
The ration between the light shift and the Raman Rabi frequency, i.e. line width,
determines the stability requirement of our laser indensity.
With cross coupling, the differential shift is (assuming $\delta\ll\Delta$),
\begin{align*}
  \Delta\delta\approx&\frac{\Omega_1^2}{4\Delta}-\frac{\Omega_1^2M_2^2}{4\Delta M_1^2}-\frac{\Omega_2^2}{4\Delta}+\frac{\Omega_2^2M_1^2}{4\Delta M_2^2}\\
  =&\frac{M_1^2-M_2^2}{4\Delta}\left(\frac{\Omega_1^2}{M_1^2}+\frac{\Omega_2^2}{M_2^2}\right)\\
  \propto&\frac{M_1^2-M_2^2}{4\Delta}\left(P_1+P_2\right)
\end{align*}
which is also minimized when $P_1=P_2$ at a given Raman Rabi frequency.

The ratio with the Raman Rabi frequency is,
\begin{align*}
  \frac{\Delta\delta}{\Omega_{Raman}}\approx&\frac{M_1^2-M_2^2}{4\Delta}\left(\frac{\Omega_1^2}{M_1^2}+\frac{\Omega_2^2}{M_2^2}\right)\frac{2\Delta}{\Omega_1\Omega_2}\\
  =&\frac{M_1^2-M_2^2}{2\Omega_1\Omega_2}\left(\frac{\Omega_1^2}{M_1^2}+\frac{\Omega_2^2}{M_2^2}\right)\\
  =&\frac{M_1^2-M_2^2}{M_1M_2}
\end{align*}
the absolute value of which is also minimized when the ratio $M_1/M_2$ is the closest to $1$.

Due to the coupling strength difference, we have $M_2\gg M_1$ in our experiment, which means,
\begin{align*}
  \left|\frac{\Delta\delta}{\Omega_{Raman}}\right|\approx&\frac{M_2}{M_1}
\end{align*}
In order to keep the resonance stable within the linewidth of the Raman resonance,
i.e. $\Omega_{Raman}$, we need to maintain a relative stability of $\Delta\delta$,
therefore relative stability of the laser power to better than $M_1/M_2$.

\section{Raman Transfer versus STIRAP}

An alternative method often used to create and prepare the internal states of ultracold molecule
is stimulated Raman adiabatic passage (STIRAP)\todo{\cite{}}.
Compared to Raman transition, which uses detuning from the excited state
to reduce scattering during the transfer, STIRAP relies on a superposition between
the initial and final state as a dark state to achieve the same goal.
The dark state in STIRAP is created due to a destructive interference of transition
from the initial and final state to the excited state.

Similar to Raman transfer, STIRAP in an ideal three-level system can achieve
full coherent transfer with arbitrarily small scattering probability
when given unlimited time and power budget.
However, in reality, states and coupling that exist outside the ideal three-level system
always have a non-zero probability of scattering loss.
In this section, we will apply the approach we took for Raman transition
and apply it to STIRAP. We will then compare the loss caused by different practical limitations
and discuss which approach should be taken under certain circumstance.

\ref{fig:raman-transfer-generic-stirap-model}

\begin{figure}
  \centering
  \includegraphics[width=0.6\textwidth]{figures/raman_transfer_generic_raman_model.pdf}
  \caption[Generic model for a real STIRAP]{
    \todo{}
    \label{fig:raman-transfer-generic-stirap-model}}
\end{figure}

\subsection{Additional Initial and Final States}

\subsection{Additional Excited states}

\subsection{Cross Coupling Between Light Addressing Initial and Final States}

\subsection{Conclusion}

\section{States Selection}
\label{ch:raman-transfer:state-selction}

(Differential Light Shift)
(Scattering)

\subsection{Excited State Selection}

\subsection{Ground States Selection}

\subsubsection{Final Molecular State}

\subsubsection{Initial Atomic State}

\section{Raman Transfer Results}

\subsection{Scaling of Raman Transition Parameters}

% \include{chapters/conclusion}

\setstretch{\dnormalspacing}

% the back matter
\begin{appendices}
  % -*- mode: latex-mode; TeX-engine: xetex; LaTeX-command-style: (("" "SOURCE_DATE_EPOCH=0 %(PDF)%(latex) --shell-escape %S%(PDFout)")); TeX-master: "../dissertation.tex"; -*-

\chapter{Computer Control Hardware Specification}
\label{appendex:computer-control}

In this appendex we list the specification of the important hardware
used in our computer control system.
See section~\ref{ch:computer-control:backend} for the integration
of these hardware into the system.

\section{FPGA}
\label{appendex:computer-control:fpga}

We use the \href{https://www.xilinx.com/products/boards-and-kits/ek-z7-zc702-g.html}{ZC702 evaluation board}~(part number EK-Z7-ZC702-G).
The on board CPU has a maximum clock speed of $666.667~\mathrm{MHz}$
and supports the VFPv3 and NEON extension for floating point and SIMD instructions.
The board also includes $1~\mathrm{GiB}$ of DDR3 RAM connected to the CPU.
The FPGA is configured to run at $100~\mathrm{MHz}$ which determines
the highest timing resolution of $10~\mathrm{ns}$ in our experiment.

We connect the FPGA to the peripherals using two FMC LPC connectors
each containing 68 pins used for single-ended signals.
Each FMC connector is used to control 11 DDS's,
which will be described in section~\ref{appendex:computer-control:fpga},
and one of the connector is also used to output 32 logical control signals
and the clock to synchronize with other devices.

The DDS's on each FMC connector are controlled using a shared parallel bus
with 7-bit address, 16-bit data and 6 control signals.
A chip select pin for each DDS is used to enable the relevant one for update.
The setup allows, in general case, one DDS on each FMC connector to be programmed
simultaneously while at the same time updating the logical outputs.
This concurrently update capability, however, is not currently used in the experiment
and only one update at a time is allowed.

\section{DDS}
\label{appendex:computer-control:fpga}

\todo{
  Part number
  Clock speed, bandwidth
  output amplifier, maximum power, noise, cross talk minimized by shielding
  Programming mode
}

\section{NI DAQ Card}

\todo{
  Part number
  Voltage range, resolution
  refresh speed
  external trigger and clock
  Ground isolation
}

\section{USRP}

\todo{
  Part number
  Daughter board, noise with higher frequency daughter board
  sampling rate/bandwidth
  output power
}

  % -*- mode: latex-mode; TeX-engine: xetex; LaTeX-command-style: (("" "SOURCE_DATE_EPOCH=0 %(PDF)%(latex) --shell-escape %S%(PDFout)")); TeX-master: "../dissertation.tex"; -*-

\chapter{Full Raman Sideband Cooling Sequence}
\label{appendex:rsc}

Each Raman pulse in the cooling sequence is followed immediately by an optical pumping pulse.
The full parameters for the Raman pulses, including the cooling ``axis'',
the sideband ``order ($\Delta n$)'', the cooling frequency ``$\delta '$",
the carrier ($\Delta n=0$) frequency ``$\delta_0'$'', the pulse ``duration'',
the pulse strength in ``$\Omega_0$'',
and the beam of which a non-uniform ``power ramp'' is applied, are listed in 6 groups below.
The applied cooling frequency, $\delta'$,
is the two-photon detuning given relative to the zero-field $F=1$ and $F=2$ hyperfine splitting
of $1.7716261288(10)$GHz~\cite{steck_sodium_nodate}.
Due to the Stark shifts of the Raman beams, the carrier transition, $\delta'_0$,
varies with the power of the Raman beams.
$\delta'_0$ is given also relative to the zero-field hyperfine splitting.
The strength of the pulses given in $\Omega_0$ determines the two-photon Rabi frequency,
$\Omega_{n,\Delta n}=\Omega_0 \langle n|e^{i \vec{k} \cdot \vec{r}}|n+\Delta n\rangle$.
We adopt the convention that a $\pi$-pulse between state $n$ and $n+\Delta n$ requires a duration $\pi/\Omega_{n,\Delta n}$.
The difference between $\delta'$ and $\delta'_0$ gives the motional sideband frequency, $\delta$.
Many Raman pulses include a ``power ramp'' with a Blackman envelope~\cite{kasevich_laser_1992} to minimize off-resonant excitations.
Because each Raman pulse is a product of two spatial- and temporal-overlapped laser beams,
the ``power ramp'' is applied only to the beam that has the smaller light shift
(we label the beam by the corresponding $F$ number) while the other beam has a square-pulse shape.
For a Raman pulse with a power ramp,
the Rabi frequency gives the arithmetic mean over the duration of the pulse.

\newpage
\subsubsection{Group 1}
This group is repeated 4 times.
\begin{center}
  \begin{tabular}{|c|c|c|c|c|c|c|}
    \hline
    Axis&$\Delta n$&$\delta'$ (MHz)&$\delta'_0$ (MHz)&Duration ($\mu$s)& $\Omega_0$ (kHz)&Power ramp\\\hline
    $x$&-2&$19.625$&$18.649$&44.1&$2\pi\times23$&F1\\\hline
    $y$&-2&$19.615$&$18.648$&28.6&$2\pi\times35$&F1\\\hline
    $x$&-1&$19.130$&$18.649$&36.9&$2\pi\times23$&F1\\\hline
    $y$&-1&$19.615$&$18.648$&24.0&$2\pi\times35$&F1\\\hline
  \end{tabular}
\end{center}

\subsubsection{Group 2}
This group is repeated 5 times.
\begin{center}
  \begin{tabular}{|c|c|c|c|c|c|c|}
    \hline
    Axis&$\Delta n$&$\delta'$ (MHz)&$\delta'_0$ (MHz)&Duration ($\mu$s)& $\Omega_0$ (kHz)&Power ramp\\\hline
    $z$&-5&$19.030$&$18.605$&81.5&$2\pi\times16$&F2\\\hline
    $x$&-2&$19.625$&$18.649$&44.1&$2\pi\times23$&F1\\\hline
    $z$&-4&$18.940$&$18.605$&76.3&$2\pi\times16$&F2\\\hline
    $y$&-2&$19.615$&$18.648$&28.6&$2\pi\times35$&F1\\\hline
    $z$&-5&$19.030$&$18.605$&81.5&$2\pi\times16$&F2\\\hline
    $x$&-1&$19.130$&$18.649$&36.9&$2\pi\times23$&F1\\\hline
    $z$&-4&$18.940$&$18.605$&76.3&$2\pi\times16$&F2\\\hline
    $y$&-1&$19.130$&$18.648$&24.0&$2\pi\times35$&F1\\\hline
  \end{tabular}
\end{center}
\subsubsection{Group 3}
This group is repeated 6 times.
\begin{center}
  \begin{tabular}{|c|c|c|c|c|c|c|}
    \hline
    Axis&$\Delta n$&$\delta'$ (MHz)&$\delta'_0$ (MHz)&Duration ($\mu$s)& $\Omega_0$ (kHz)&Power ramp\\\hline
    $z$&-4&$18.940$&$18.605$&76.3&$2\pi\times16$&F2\\\hline
    $x$&-2&$19.625$&$18.649$&44.1&$2\pi\times23$&F1\\\hline
    $z$&-3&$18.858$&$18.605$&70.2&$2\pi\times16$&F2\\\hline
    $y$&-2&$19.615$&$18.648$&28.6&$2\pi\times35$&F1\\\hline
    $z$&-4&$18.940$&$18.605$&76.3&$2\pi\times16$&F2\\\hline
    $x$&-1&$19.130$&$18.649$&36.9&$2\pi\times23$&F1\\\hline
    $z$&-3&$18.858$&$18.605$&70.2&$2\pi\times16$&F2\\\hline
    $y$&-1&$19.130$&$18.648$&24.0&$2\pi\times35$&F1\\\hline
  \end{tabular}
\end{center}

\newpage
\subsubsection{Group 4}
This group is repeated 7 times.
\begin{center}
  \begin{tabular}{|c|c|c|c|c|c|c|}
    \hline
    Axis&$\Delta n$&$\delta'$ (MHz)&$\delta'_0$ (MHz)&Duration ($\mu$s)& $\Omega_0$ (kHz)&Power ramp\\\hline
    $z$&-3&$18.858$&$18.605$&70.2&$2\pi\times16$&F2\\\hline
    $x$&-2&$19.625$&$18.649$&44.1&$2\pi\times23$&F1\\\hline
    $z$&-2&$18.773$&$18.605$&62.7&$2\pi\times16$&F2\\\hline
    $y$&-2&$19.615$&$18.648$&28.6&$2\pi\times35$&F1\\\hline
    $z$&-3&$18.858$&$18.605$&70.2&$2\pi\times16$&F2\\\hline
    $x$&-1&$19.130$&$18.649$&36.9&$2\pi\times23$&F1\\\hline
    $z$&-2&$18.773$&$18.605$&62.7&$2\pi\times16$&F2\\\hline
    $y$&-1&$19.130$&$18.648$&24.0&$2\pi\times35$&F1\\\hline
  \end{tabular}
\end{center}

\newpage
\subsubsection{Group 5}
This group is repeated 10 times.
\begin{center}
  \begin{tabular}{|c|c|c|c|c|c|c|}
    \hline
    Axis&$\Delta n$&$\delta'$ (MHz)&$\delta'_0$ (MHz)&Duration ($\mu$s)& $\Omega_0$ (kHz)&Power ramp\\\hline
    $z$&-2&$18.773$&$18.605$&62.7&$2\pi\times16$&F2\\\hline
    $x$&-1&$19.130$&$18.649$&36.9&$2\pi\times23$&F1\\\hline
    $z$&-1&$18.685$&$18.605$&52.5&$2\pi\times16$&F2\\\hline
    $y$&-1&$19.130$&$18.648$&24.0&$2\pi\times35$&F1\\\hline
    $z$&-2&$18.773$&$18.605$&62.7&$2\pi\times16$&F2\\\hline
    $x$&-1&$19.130$&$18.649$&70.0&$2\pi\times23$&F1\\\hline
    $z$&-1&$18.685$&$18.605$&52.5&$2\pi\times16$&F2\\\hline
    $y$&-1&$19.130$&$18.648$&46.0&$2\pi\times35$&F1\\\hline
  \end{tabular}
\end{center}

\newpage
\subsubsection{Group 6}
This group is repeated 30 times.
\begin{center}
  \begin{tabular}{|c|c|c|c|c|c|c|}
    \hline
    Axis&$\Delta n$&$\delta'$ (MHz)&$\delta'_0$ (MHz)&Duration ($\mu$s)& $\Omega_0$ (kHz)&Power ramp\\\hline
    $z$&-1&$18.683$&$18.605$&78.7&$2\pi\times11$&F2\\\hline
    $z$&-1&$18.683$&$18.605$&135.0&$2\pi\times11$&F2\\\hline
    $z$&-1&$18.685$&$18.605$&78.7&$2\pi\times11$&F2\\\hline
    $x$&-1&$19.130$&$18.649$&36.9&$2\pi\times23$&F1\\\hline
    $y$&-1&$19.130$&$18.648$&24.0&$2\pi\times35$&F1\\\hline
    $z$&-1&$18.685$&$18.605$&78.7&$2\pi\times11$&F2\\\hline
    $z$&-1&$18.685$&$18.605$&135.0&$2\pi\times11$&F2\\\hline
    $z$&-1&$18.685$&$18.605$&78.7&$2\pi\times11$&F2\\\hline
    $x$&-1&$19.130$&$18.649$&70.0&$2\pi\times23$&F1\\\hline
    $y$&-1&$19.130$&$18.648$&46.0&$2\pi\times35$&F1\\\hline
  \end{tabular}
\end{center}

\end{appendices}
% -*- mode: latex; TeX-engine: xetex; LaTeX-command-style: (("" "SOURCE_DATE_EPOCH=0 %(PDF)%(latex) --shell-escape %S%(PDFout)")); TeX-master: "../dissertation.tex"; -*-

\clearpage
\begin{spacing}{\dcompressedspacing}
  \bibliography{references}
  \addcontentsline{toc}{chapter}{References}
  \bibliographystyle{osajnl}
  % \bibliographystyle{apalike2}
\end{spacing}

\end{document}
